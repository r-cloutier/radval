\documentclass[twocolumn]{emulateapj}
\usepackage[bookmarks,bookmarksopen,colorlinks,linkcolor={blue},
  citecolor={blue},urlcolor={blue}]{hyperref}
\usepackage{amsmath}
\usepackage{natbib}
\usepackage{color}
\usepackage[x11names]{xcolor}
%\usepackage{geometry}
%\usepackage{pdflscape}
\usepackage{graphics}
\usepackage{enumerate}
\usepackage{setspace}
%\usepackage{siunitx}
\newcommand{\gaia}[1]{\emph{Gaia}#1}
\newcommand{\kepler}[1]{\emph{Kepler}#1}
\newcommand{\ktwo}[1]{\emph{K2}#1}
\newcommand{\tess}[1]{\emph{TESS}#1}
\newcommand{\teff}[1]{$T_{\text{eff}}$#1}
\newcommand{\logg}[1]{$\log{g}$#1}

\defcitealias{fulton17}{F17}

% Multiple layer commands
\newcommand{\cdbox}[1]{%
  \colorlet{currentcolor}{.}%
  {\color{Blue1}%
    \dbox{\color{currentcolor}#1}}%
}
\usepackage{dashbox,framed,color,ocg-p}
\fboxsep=1pt
\fboxrule=1pt
\newcommand{\ToggleLayer}[2]{%
  % #1: layer name,
  % #2: link text
  \leavevmode
  \pdfstartlink user {
    /Subtype /Link
    /Border [0 0 0]%
    /A <<
      /S/JavaScript
      /JS (
         var aOCGs = this.getOCGs(), Layer;
         var Layers = "#1".split(","), Active = -1, i, l;
         for (l=0; l<Layers.length; l++) {
           Layer = Layers[l];
           for (i=0; aOCGs && i<aOCGs.length; i++) {
             if (aOCGs[i].state && aOCGs[i].name == Layer) {
               Active = l;
               aOCGs[i].state = false;
             }
           }
           if (Active >= 0) break;
         }
         if (Active == -1) {
           for (l=0; l<Layers.length; l++) {
             if (Layers[l] == "") Active = l;
           }
         }
         Active = Active + 1;
         if (Active == Layers.length) Active = 0;
         Layer = Layers[Active];
         for (i=0; aOCGs && i<aOCGs.length; i++) {
           if (aOCGs[i].name == Layer) aOCGs[i].state = true;
         }
      )
    >>
  }#2%
  \pdfendlink
}

\shortauthors{Cloutier et al.}
\shorttitle{Evolution of the radius valley with stellar mass}

\begin{document}
\title{Evolution of the Occurrence Rate of small close-in planets around low mass dwarf stars from \emph{KEPLER} and \emph{K2}}
\author{Ryan Cloutier\altaffilmark{1,2} et al.}

\email{cloutier@cfa.harvard.edu}

\altaffiltext{1}{Center for Astrophysics $|$ Harvard \& Smithsonian,
  60 Garden Street, Cambridge, MA, 02138, USA}
\altaffiltext{2}{Dept. of Astronomy \& Astrophysics, University
of Toronto, 50 St. George Street, Toronto, ON, M5S 3H4, Canada}

\begin{abstract}
  Recent observational studies have revealed a prominent gap in the occurrence rate
  of close-in planet radii around Sun-like stars.
  Resolving the so-called radius valley around low mass stars can provide valuable
  constraints on the physical mechanisms that sculpt the valley and so far have been largely limited
  by relatively poor counting statistics.
  Here we calculate the occurrence rate of small close-in planets around low mass dwarf stars
  using the known planet populations from both the primary \kepler{} and \ktwo{} missions while
  exploiting the precise \gaia{} DR2 data to refine the stellar radii and masses. 
  %The empirical planet population is corrected using the published completeness and reliability
  %products for \kepler{} stars which we supplement by conducting
  %injection/recovery tests of transiting planetary signals using our own pipeline to characterize
  %our completeness for the \ktwo{} stars in our sample.
  The application of appropriate completeness corrections to the empirical planet population
  clearly reveals the radius valley in the maximum a-posteriori occurrence rate map as a
  function of orbital period and planet radius. The measured slope of the valley with orbital separation
  is shown to differ in sign from the slope measured around Sun-like stars which suggests that
  distinct physical processes dominate the formation and evolution of planets in each stellar mass regime.
  We also show that the prominence and location of radius valley features evolve with stellar mass as
  the relative occurrence of terrestrial to gaseous planets increases by an order of magnitude from mid-K to
  mid-M dwarfs while the valley features evolve to smaller
  planet sizes with decreasing stellar mass in agreement with physical models of photoevaporation,
  gas-poor formation, and core-powered mass loss. Although current measurements are
  insufficient to robustly identify any physical model as the dominant formation pathway of the radius valley,
  we argue that robust inferences may be obtained by \tess{} with $\mathcal{O}(\mathbf{?})$
  mid-M dwarfs observed with 2-minute cadence.
\end{abstract}


\section{Introduction}
NASA's \kepler{} space telescope has discovered thousands of exoplanets over its lifetime and
consequently enabled robust investigations of the occurrence rate of planets within our galaxy.
One striking outcome of such studies was that the so-called super-Earths and sub-Neptunes---whose
radii span sizes intermediate between those of the Earth and Neptune---represent the most common
type of planet around Sun-like stars and early M dwarfs alike
\citep[e.g.][]{youdin11,howard12,dressing13,fressin13,petigura13b,morton14,dressing15a,mulders15,gaidos16,fulton17,hardegree19}.
Furthermore, mass measurements of many of these transiting planets via transit-timing variations
or precision radial velocity measurements revealed that the majority of planets
smaller than $\sim 1.6$ R$_{\oplus}$ are consistent with having bulk terrestrial compositions
\citep[e.g.][]{weiss14,dressing15b,rogers15}.

Early studies of the \kepler{} planet population
hinted that planets at small orbital separations exhibited a
bimodal radius distribution \citep[e.g.][]{owen13}---commonly referred to as the radius valley---that
is thought to be representative of a population of small, predominantly rocky planets plus a population
of inflated gaseous planets that have retained significant H/He envelopes.
Consequently, numerous studies of planet formation and evolution sought to explain the
apparent bimodality. One such proposed mechanism is
that of photoevaporation wherein the gaseous envelopes of small close-in planets may be stripped by
X-ray and extreme ultraviolet (XUV) radiation from their host stars during the first $\sim 100$ Myrs
of the planet's lifetime
\citep{jackson12,owen13,jin14,lopez14,chen16,owen17,jin18,lopez18}. Another possible explanation
invokes gas-poor formation wherein gas accretion is delayed by dynamical friction whilst the
planetary core is still embedded within the protoplanetary disk until a point at which the gaseous disk
has almost completely dissipated after just a few Myrs \citep{lee14,lee16,lopez18}. More recently,
the radius valley may also be explained by core-powered mass loss wherein the
luminosity from a planetary core's primordial energy reservoir from formation drives atmospheric escape
over Gyr timescales \citep{ginzburg18,gupta19a,gupta19b}.

Observational tests of the aforementioned theoretical frameworks have become feasible in recent years due to 
the precise refinement of measured planet radii following improved stellar host characterization via  
spectroscopy, asteroseismology, and \gaia{} parallaxes
\citep[e.g.][]{fulton17,berger18,fulton18,vaneylen18,martinez19}. Each of these independent studies clearly
resolved the radius valley among small close-in planets orbiting Sun-like stars.
A variety of trends were also observed in either
the raw or in the completeness-corrected (i.e. the occurrence rate) distributions of close-in planets. Firstly,
the location of the radius valley around FGK stars is period-dependent with slope
$\mathrm{d}\log{r_p} / \mathrm{d}\log{P} \sim -0.1$ \citep{vaneylen18,martinez19}, a result that is consistent
with both photoevaporation and core-powered mass loss models but is inconsistent with the late formation of
terrestrial planets in a gas-poor environment. Secondly, the feature locations (i.e. the weighted
average radius of the peaks and valley) appear to exist at smaller planet radii with decreasing stellar
mass \citep{fulton18,wu19}.

In this study, we extend the investigation of the occurrence rate of small close-in planets to systems hosted by
low mass dwarf stars later than mid-K dwarfs.
The empirical population of known planets in this stellar mass regime features nearly an order of magnitude
fewer planets than around Sun-like stars, thus making the detection of the radius valley around low mass stars more
difficult and at a lower signal-to-noise. This fact is clearly evidenced in the empirical \kepler{} planet population
for which the radius valley around Sun-like stars (\teff{} $\in [4700,6500]$ K) is clearly exhibited whereas a similar
feature around low mass stars (\teff{} $< 4700$ K) is not easily discernible by-eye (Fig.~\ref{fig:berger} based on
the data from \citealt{berger18}). This study leverages the precise stellar parallaxes from the \gaia{} DR2
for low mass stars observed by \kepler{} and \ktwo{} to refine the stellar parameters and compute precise
occurrence rates of close-in planets with the goal of resolving the radius valley and accurately measuring the
location of the radius valley features and their uncertainties. Although it is unlikely that a single physical
mechanism is responsible for sculpting the radius valley, investigation the evolution of the valley features with
stellar mass can allude to which process---if any---dominates the evolution of close-in planets.


\begin{figure}
  \centering
  \includegraphics[width=0.98\hsize]{figures/Bergerplanethist.png}
  \caption{Empirical distributions of \kepler{} planet radii. Histograms of \kepler{} planet radii
    from \cite{berger18} for planets with host stellar effective temperatures \teff{} $\in [4700,6500]$ K
    (\emph{blue}) and \teff{} $<4700$ K (\emph{red}). The former subset of 2816 planets corresponds to the
    effective temperature range considered in the California Kepler Survey \citep[CKS;][]{fulton17}
    wherein the radius valley is clearly
    resolved in the empirical distribution even without completeness corrections. A similar bimodal
    structure is not resolved in the empirical distribution of the latter subset around low mass stars
    due in-part to the relatively poor counting statistics with just 350 planets.}
  \label{fig:berger}
\end{figure}


In Sects.~\ref{sect:stars} and ~\ref{sect:planets} we define our stellar sample from \kepler{} and \ktwo{}
and compile our sample of confirmed planets from each mission.
In Sect.~\ref{sect:completeness} we derive the transiting planet detection completeness and use those results
to calculate the occurrence rate of small close-in planets in which the structure of the radius valley around
low mass stars is resolved (Sect.~\ref{sect:occurrence}). Our results as a function stellar mass are compared to
model predictions in Sect.~\ref{sect:models}. We conclude 
with a discussion of our results and its implications in Sect.~\ref{sect:conclusion}.


\section{Low Mass Dwarf Stellar Sample} \label{sect:stars}
The goal of this study is to extend measurements of the occurrence rate of close-in planets to planetary systems hosted
by low mass dwarf stars with effective temperatures \teff{} $<4700$ K: the lower limit of \teff{}
considered in the California Kepler Survey \citep[CKS;][]{fulton17}.
This adopted temperature threshold approximately corresponds to spectral types later than
K3.5V \citep{pecaut13}. In the following subsections we define our stellar sample from both \kepler{} or \ktwo{.}

\subsection{Kepler Stellar Sample} \label{sect:kep}
Following the release of \gaia{} DR2 \citep{lindegren18}, \cite{berger18} cross-matched \kepler{} target stars
with DR2 and compiled a catalog of stellar
parallaxes $\varpi$, 2MASS $K_s$-band magnitudes, and spectroscopic measurements of \teff{,} \logg{,} and [Fe/H]
for $\sim 178,000$ stars observed as part of the primary \kepler{} mission. Spectroscopic measurements were obtained from
either the DR25
Kepler Stellar Properties Catalog \citep[KSPC;][]{mathur17}, the California
Kepler Survey \citep[CKS;][]{petigura17} where available, and \teff{} values for stars with \teff{} $<4000$ K were compiled from
\cite{gaidos16}. The full set of available stellar parameters were used as input within the spectral classification code
\texttt{isoclassify} \citep{huber17} to calculate stellar luminosities. The resulting luminosity values were consequently combined
with \teff{} measurements to refine the stellar radii using the Stefan-Boltzmann law for the majority of \kepler{} FGK stars.
However, bolometric corrections for \kepler{} M dwarfs with \teff{} $<4100$ K
and absolute $K_s$-band magnitudes $M_{K_s}>3$ are known to suffer significant inaccuracies owing to incomplete
molecular line lists. For these stars, \cite{berger18} instead adopted the empirically-derived M dwarf radius-luminosity
relation from \cite{mann15} to refine the M dwarf stellar radii. \cite{berger18} also combined the \teff{} 
luminosity measurements to derive stellar evolutionary flags aimed at classifying stars as either a dwarf, a subgiant, or a
red giant.

Stellar masses $M_s$ are not reported by \cite{berger18}. In order to study the \kepler{} planet population as a function of $M_s$,
we derive $M_s$ values given the measured stellar radii $R_s$ using the mass-radius relation from \cite{boyajian12} which is
applicable to both K and M dwarfs.
\cite{boyajian12} acquired interferometric measurements with the \emph{CHARA} array of 21 nearby K and M dwarfs
to measure the angular size of each stellar disk at the level of $\lesssim 5$\%. Their stellar sample was supplemented by 12
literature measurements of $R_s$ from interferometry. Mass measurements were then derived using the $K_s$-band mass-luminosity
relation from \cite{henry93} which was valid for their full stellar sample spanning 0.13-0.90 R$_{\odot}$. \cite{boyajian12}
parameterized the stellar mass-radius relationship as a quadratic in $M_s$ and reported values and uncertainties for each polynomial
coefficient. Here, we assume independent Gaussian probability density functions (PDF) for each coefficient and sample their values
along with each star's $R_s$ from their respective measurement uncertainties to derive the $M_s$ PDF for all of the low mass dwarfs
in our preliminary \kepler{} sample.

We define our final \kepler{} stellar sample by focusing on stars that satisfy the following criteria:

\begin{enumerate}
\item \kepler{} magnitude $K_p < 16$,
\item $T_{\text{eff}} - \sigma_{T_{\text{eff}}} \leq 4700$ K,
\item $R_s - \sigma_{R_s} \leq 0.8$ R$_{\odot}$,
\item $M_s - \sigma_{M_s} \leq 0.8$ M$_{\odot}$, and
\item and an evolutionary flag corresponding to a dwarf star. 
\end{enumerate}

\noindent Based on these criteria, we retrieve 3965 low mass \kepler{} stars whose
stellar parameters are depicted in Fig.~\ref{fig:stars}.
In our \kepler{} sample, the \kepler{} magnitudes span $K_p \in [10.35, 16.00]$ with a median value of 15.16,
effective temperatures span \teff{} $\in [3154, 4870]$ K with a median value of 4394 K,
stellar radii span $R_s \in [0.17, 0.87]$ R$_{\odot}$ with a median value of 0.68 R$_{\odot}$, and
stellar masses span $M_s \in [0.13, 0.88]$ M$_{\odot}$ with a median value of 0.70 M$_{\odot}$.
Our final \kepler{} sample boasts a median fractional $R_s$ uncertainty of $\sim 6.7$\% which is $\sim 4-5$
times smaller than the typical $R_s$ uncertainty reported in the KSPC. The median fractional uncertainty on
$M_s$ is $\sim 5.5$\%.

\begin{figure}
  \centering
  \includegraphics[width=0.98\hsize]{figures/stellar_corner_KepandK2.png}
  \caption{Low mass dwarf stellar samples from \kepler{} and \ktwo{.} Distributions of \kepler{} magnitudes,
    effective temperatures, stellar radii, and stellar masses for stars in our final stellar sample from either
    \kepler{} (\emph{blue histogram and markers}) or \ktwo{} (\emph{red histogram and markers}).}
  \label{fig:stars}
\end{figure}

\subsection{K2 Stellar Sample}
We first retrieved the list of probable low mass dwarf stars observed in any \ktwo{} campaign by querying
MAST\footnote{Mikulski Archive for Space Telescopes, \url{https://archive.stsci.edu/k2/}.}. Our initial
search was restricted to \ktwo{} stars with \teff{} $<4900$ K, \logg{} $>4$, and $R_s<1$ R$_{\odot}$. Note that these
criteria are not intended to represent the parameter ranges for low mass dwarf stars but are intended as
conservative conditions to encapsulate all such stars prior to their refinement using the \gaia{} DR2
data. From MAST we retrieve each star's Ecliptic Plane Input Catalog
(EPIC) numerical identifier, stellar photometry in the \kepler{} bandpass $K_p$ and 2MASS bands $JHK_s$, along
with measured values of \teff{,} \logg{,} [Fe/H], and $R_s$.

We proceed with refining the stellar parameters by cross-matching our initial \ktwo{} sample with \gaia{}  
DR2 using the \gaia{-}\ktwo{} data products from Megan Bedell\footnote{\url{https://gaia-kepler.fun/}}. Where
available, we retrieve reach star's celestial coordinates, stellar parallaxes $\varpi$, and \gaia{} photometry.
Measurements of $R_s$ then follow from the methodology of \cite{berger18} and outlined as follows.
The formalism of \cite{bailerjones18} is used to transform the assumed
Gaussian-distributed $\varpi$ PDFs into stellar distance PDFs which need not remain Gaussian.
Using the measured distances $d$ and celestial coordinates, we interpolate over the $E_{B-V}$ extinction maps using the
\texttt{mwdust} software \citep{bovy16} to derive both the $V$ and $K_s$-band extinction coefficients $A_V$ and
$A_{K_s}$. We then calculate each star's absolute $K_s$-band magnitude $M_{K_s} = K_s - \mu - A_{K_s}$ where
the distance modulus is $\mu = 5\log_{10}(d/10\text{ pc})$.

For the earliest stars in our sample ($M_{K_s}\leq 4.6$) , for which the bolometric corrections
are still reliable, we interpolate the MIST bolometric correction grids \citep{choi16} over \teff{,} \logg{,} [Fe/H],
and $A_V$ to derive the $K_s$-band bolometric corrections $BC_{K_s}$. We then compute the
absolute bolometric magnitudes $M_{\text{bol}}=M_{K_s} + BC_{K_s}$ and consequently the bolometric stellar
luminosities as 

\begin{equation}
  L_{\text{bol}} = L_0 \cdot 10^{-0.4 M_{\text{bol}}},
\end{equation}

\noindent where $L_0 = 3.0128 \times 10^{28}$ W \citep{mamajek15}. The refined $R_s$ values
are then calculated using the Stefan-Boltzmann law given $L_{\text{bol}}$ and \teff{} with measurement uncertainties
propagated throughout.

For the remaining late type stars with $M_{K_s}>4.6$, we revert to the empirically-derived radius-luminosity relation
from \cite{mann15} to calculate the M dwarf stellar radii. \cite{mann15} fit a second-order polynomial to $R_s$ as
a function of $M_{K_s}$ which has a characteristic dispersion in the fractional radius uncertainty of 2.89\%. To quantify
the final $R_s$ uncertainty we sample $M_{K_s}$ from its posterior PDF and transform each $M_{K_s}$ draw to an $R_s$ value
using the aforementioned radius-luminosity relation. To each star's derived $R_s$ PDF, we add an additional
dispersion term, in quadrature, whose fractional uncertainty is 2.89\%. Stellar masses within our \ktwo{} sample are derived
identically to the method applied to the \kepler{} sample using the \cite{boyajian12} stellar
mass-radius relation (see Sect.~\ref{sect:kep}). 

We define our final \ktwo{} stellar sample of low mass dwarf stars similarly to our definition of the \kepler{} sample.
Explicitly, we focus on stars that obey the following criteria:

\begin{enumerate}
\item $K_p < 14.7$,
\item $T_{\text{eff}} - \sigma_{T_{\text{eff}}} \leq 4700$ K,
\item $R_s - \sigma_{R_s} \leq 0.8$ R$_{\odot}$,
\item $M_s - \sigma_{M_s} \leq 0.8$ M$_{\odot}$, and
\item $R_s < R_{s,\text{max}}$.
\end{enumerate}

\noindent Because our \ktwo{} sample lacks any evolutionary flags, we adopt the following ad hoc upper limit on $R_s$
from \cite{fulton17} that aims to reject evolved stars:

\begin{equation}
  R_{s,\text{max}} = \text{R}_{\odot} \cdot 10^{0.00025(T_{\text{eff}}/\text{K}-5500)+0.2}.
\end{equation}

\noindent Based on these criteria, we retrieve 13428 low mass \ktwo{} stars whose
stellar parameters are also depicted in Fig.~\ref{fig:stars}.
In our \ktwo{} sample, the \kepler{} magnitudes span $K_p \in [8.47, 14.68]$ with a median value of 14.04,
effective temperatures span \teff{} $\in [3246, 4856]$ K with a median value of 4017 K,
stellar radii span $R_s \in [0.14, 0.94]$ R$_{\odot}$ with a median value of 0.70 R$_{\odot}$, and
stellar masses span $M_s \in [0.09, 0.93]$ M$_{\odot}$ with a median value of 0.69 M$_{\odot}$.
The stars in this sample exhibit a median fractional $R_s$ uncertainty of $\sim 3.5$\% which is $\sim 2$
times smaller than the typical $R_s$ uncertainty obtained for stars in our \kepler{} sample.
The median fractional uncertainty on $M_s$ is $\sim 3.9$\%.

Each of the \kepler{} and \ktwo{} stellar samples are dominated by mid-to-late K dwarfs
with temperatures and radii $\gtrsim 3800$ K and $\gtrsim 0.6$ R$_{\odot}$ respectively. This
fact will have important implications on our ability to precisely measure the planet occurrence
rate around the lowest mass dwarf stars in our sample.


\section{Population of Small Close-in Planets around Low Mass Dwarf Stars} \label{sect:planets}
Here we define the population of small close-in planets orbiting stars contained in our stellar sample.
Our initial sample of transiting planets from either \kepler{} or \ktwo{} were retrieved from the
NASA Exoplanet Archive \citep{akeson13} on June 15, 2019. Only confirmed
planets---based on their Exoplanet Archive dispositions---with orbital periods
$P\in [0.5,100]$ days are included. By considering confirmed
planets only we naturally focus on a subset of the true empirical population of small close-in planets
without being contaminated by various astrophysical false positive scenarios that may plague the planet
candidates excluded from our initial sample.

The refined stellar radii derived in Sect.~\ref{sect:stars} enable us to derive more precise
planetary radii.
We refine the planetary radii $r_p$ by retrieving point estimates of each planet's scaled planetary radius
$r_p/R_s$ which often includes a median value 
accompanied by the 16$^{\text{th}}$ and 84$^{\text{th}}$ percentiles. In cases for which the $r_p/R_s$ uncertainties
are symmetric, we assume that the $r_p/R_s$ posterior PDF is Gaussian. For planets with asymmetric reported
uncertainties, we fit the $r_p/R_s$ percentiles with a skew-normal distribution using the
\texttt{scipy.skewnorm python} class. We fit for the location, scale, and shape parameters of the
distribution such that its resulting percentiles are consistent with
the $r_p/R_s$ point estimates reported for each planet. The refined planetary radii are then derived by sampling the
fitted $r_p/R_s$ and $R_s$ distributions. We then update our planet sample by
only considering planets whose radii are consistent with $r_p = 0.5-4$ R$_{\oplus}$.

From the distributions of $R_s$, \teff{,} $M_s$, and $P$ or each planet and host star, we derive the planets'
semimajor axes $a$ and insolations $F$ via

\begin{equation}
  \frac{F}{F_{\oplus}} = \left( \frac{R_s}{R_{\odot}} \right)^2  \left( \frac{T_{\text{eff}}}{5777 \text{ K}} \right)^4 \left( \frac{a}{1 \text{ AU}} \right)^{-2}.
\end{equation}

Our final sample of confirmed small close-in planets 
contains 275 \kepler{} and 53 \ktwo{} planets respectively. Their respective median fractional
radius uncertainties are 7.1\% and 9.0\%. Properties of the 328 confirmed planets in our sample
are reported in Tables~\ref{table:planetsKep} and~\ref{table:planetsK2}. Our planet sample is
depicted in Fig.~\ref{fig:Ndet} as two-dimensional maps of the number of planet detections in
the period-radius and insolation-radius spaces. The two-dimensional histogram maps are computed by
Monte-Carlo sampling planets from their $F$ and $r_p$
measurement uncertainties and with a fractional precision on $P$ inflated to 20\%.

\begin{figure*}
  \centering
  \includegraphics[width=0.98\hsize]{figures/Ndetmap_Msgt0d0_xbin45_ybin27.png}
  \caption{Empirical population of confirmed close-in planets around low mass stars.
    The distribution of 275 and 53 confirmed planets from \kepler{}
    and \ktwo{} respectively as a function of orbital period, insolation, and planet radius. The two-dimensional maps are
    Monte-Carlo sampled from the measurement uncertainties on the planetary radii and insolations while 
    the fractional uncertainties on the orbital periods are inflated to
    20\%.} %\emph{Lower panels}: same as the upper panels except for an expanded planet population that is supplemented by
    %119 planet candidates from \ktwo{} campaigns 0-8 \citep{kruse19}.}
  \label{fig:Ndet}
\end{figure*}

\capstartfalse
\begin{deluxetable*}{ccccccccc}
\tabletypesize{\small}
\tablecaption{Kepler confirmed planet parameters\label{table:planetsKep}}
\tablehead{KIC & Planet & $P$ & $F$ & $F$ upper limit & $F$ lower limit & $r_p$ & $r_p$ upper limit & $r_p$ lower limit \\
& name & [days] & [F$_{\oplus}$] & [F$_{\oplus}$] & [F$_{\oplus}$] & [R$_{\oplus}$] & [R$_{\oplus}$] & [R$_{\oplus}$]}
\startdata
2556650 & Kepler-1124 b & 2.85235 & 46.5 & 4.7 & 4.6 & 1.97 & 0.08 & 0.10 \\
2715135 & Kepler-753 b & 5.74771 & 40.2 & 4.6 & 4.5 & 1.89 & 0.30 & 0.12 \\
3234598 & Kepler-383 b & 12.90468 & 20.2 & 2.8 & 2.5 & 1.54 & 0.30 & 0.17 \\
3234598 & Kepler-383 c & 31.20122 & 6.2 & 0.8 & 0.8 & 1.49 & 0.34 & 0.22 \\
3426367 & Kepler-1308 b & 2.10434 & 55.3 & 5.6 & 5.1 & 0.89 & 0.03 & 0.14
%%3444588 & Kepler-787 b & 0.92831 & 455.5 & 59.4 & 51.4 & 1.34 & 0.09 & 0.08 \\
%%3546060 & Kepler-1110 b & 9.69311 & 38.1 & 8.3 & 7.2 & 2.44 & 0.51 & 0.20 \\
%%3554031 & Kepler-415 c & 8.70798 & 24.8 & 3.5 & 2.8 & 2.25 & 0.16 & 0.12 \\
%%3554031 & Kepler-415 b & 4.17634 & 66.3 & 8.3 & 8.0 & 1.38 & 0.11 & 0.11 \\
%%3642335 & Kepler-1410 b & 60.86610 & 0.8 & 0.1 & 0.1 & 1.38 & 0.11 & 0.11 \\
%%3728432 & Kepler-1526 b & 3.90863 & 94.2 & 20.1 & 16.8 & 1.94 & 0.17 & 0.14 \\
%%3733628 & Kepler-543 b & 13.89963 & 21.9 & 2.6 & 2.3 & 2.54 & 0.11 & 0.30 \\
%%3859079 & Kepler-786 b & 53.52927 & 3.8 & 0.4 & 0.4 & 2.30 & 0.12 & 0.14 \\
%%3966801 & Kepler-577 b & 25.69581 & 4.3 & 0.4 & 0.4 & 2.66 & 0.27 & 0.16 \\
%%4056616 & Kepler-1053 b & 2.41435 & 150.6 & 17.6 & 16.0 & 0.82 & 0.05 & 0.04 \\
%%4139816 & Kepler-235 b & 3.34022 & 57.7 & 6.1 & 5.7 & 2.50 & 0.25 & 0.25 \\
%%4139816 & Kepler-235 c & 7.82502 & 18.4 & 1.9 & 1.8 & 1.48 & 0.27 & 0.27 \\
%%4139816 & Kepler-235 d & 20.06036 & 5.3 & 0.5 & 0.5 & 2.42 & 0.40 & 0.11 \\
%%4139816 & Kepler-235 e & 46.18420 & 1.7 & 0.2 & 0.2 & 2.25 & 0.11 & 0.11 \\
%%4249725 & Kepler-120 b & 6.31251 & 36.2 & 3.9 & 3.6 & 2.55 & 0.10 & 0.07 \\
%%4249725 & Kepler-120 c & 12.79455 & 14.1 & 1.5 & 1.4 & 2.04 & 0.07 & 0.07 \\
%%4263293 & Kepler-331 d & 32.13400 & 5.0 & 1.0 & 0.9 & 2.71 & 0.23 & 0.23 \\
%%4263293 & Kepler-331 b & 8.45747 & 29.6 & 6.5 & 5.5 & 2.78 & 0.21 & 0.21 \\
%%4263293 & Kepler-331 c & 17.28117 & 11.6 & 2.7 & 2.0 & 3.02 & 0.23 & 0.23 \\
%%4633570 & Kepler-158 b & 16.70921 & 13.4 & 1.5 & 1.4 & 2.18 & 0.10 & 0.36 \\
%%4633570 & Kepler-158 c & 28.55158 & 6.5 & 0.7 & 0.8 & 1.90 & 0.33 & 0.33 \\
%%4725681 & Kepler-236 b & 8.29562 & 14.0 & 1.5 & 1.3 & 2.13 & 0.41 & 0.41 \\
%%4725681 & Kepler-236 c & 23.96794 & 3.4 & 0.4 & 0.3 & 2.10 & 0.07 & 0.12 \\
%%4736569 & Kepler-1042 b & 10.13201 & 25.1 & 5.8 & 4.6 & 1.69 & 0.12 & 0.13 \\
%%4832837 & Kepler-621 b & 2.62812 & 69.8 & 35.4 & 22.5 & 1.82 & 0.31 & 0.30 \\
%%4852528 & Kepler-80 d & 3.07215 & 145.6 & 17.3 & 17.1 & 1.56 & 0.08 & 0.08 \\
%%4852528 & Kepler-80 c & 9.52164 & 32.0 & 4.2 & 3.5 & 2.59 & 0.12 & 0.12 \\
%%4852528 & Kepler-80 e & 4.64538 & 84.0 & 11.0 & 9.5 & 1.60 & 0.08 & 0.08 \\
%%4852528 & Kepler-80 f & 0.98679 & 662.4 & 84.9 & 75.9 & 1.43 & 0.07 & 0.07 \\
%%4852528 & Kepler-80 b & 7.05353 & 48.3 & 6.4 & 5.3 & 2.59 & 0.12 & 0.11 \\
%%4913852 & Kepler-691 b & 8.11437 & 12.2 & 1.3 & 1.2 & 2.32 & 0.38 & 0.12 \\
%%4917596 & Kepler-1032 b & 3.29011 & 138.4 & 30.4 & 29.5 & 2.00 & 0.20 & 0.17 \\
%%5080636 & Kepler-974 b & 4.19450 & 28.1 & 2.9 & 2.7 & 1.64 & 0.06 & 0.24 \\
%%5175986 & Kepler-1320 b & 0.86839 & 752.8 & 156.1 & 144.9 & 1.32 & 0.12 & 0.11 \\
%%5184911 & Kepler-1324 b & 4.11586 & 106.4 & 22.7 & 19.4 & 1.55 & 0.39 & 0.25 \\
%%5185897 & Kepler-398 c & 11.41942 & 20.3 & 2.4 & 2.2 & 1.02 & 0.21 & 0.21 \\
%%5185897 & Kepler-398 b & 4.08141 & 80.0 & 9.3 & 8.9 & 1.03 & 0.06 & 0.18 \\
%%5185897 & Kepler-398 d & 6.83441 & 40.4 & 4.7 & 4.8 & 1.04 & 0.19 & 0.19 \\
%%5209845 & Kepler-1378 b & 11.95402 & 15.5 & 3.3 & 2.9 & 1.91 & 0.67 & 0.26 \\
%%5269467 & Kepler-925 b & 33.86778 & 5.6 & 1.2 & 1.0 & 2.33 & 0.46 & 0.34 \\
%%5340644 & Kepler-580 b & 8.22241 & 23.3 & 3.2 & 2.6 & 2.34 & 0.12 & 0.11 \\
%%5364071 & Kepler-49 e & 18.59612 & 5.4 & 0.6 & 0.5 & 1.77 & 0.08 & 0.08 \\
%%5364071 & Kepler-49 c & 10.91274 & 11.0 & 1.1 & 1.1 & 2.47 & 0.23 & 0.15 \\
%%5364071 & Kepler-49 b & 7.20386 & 19.2 & 2.1 & 1.9 & 2.61 & 0.08 & 0.08 \\
%%5364071 & Kepler-49 d & 2.57657 & 75.5 & 7.8 & 7.5 & 1.82 & 0.06 & 0.06 \\
%%5371776 & Kepler-304 b & 3.29570 & 157.5 & 18.5 & 16.6 & 2.68 & 0.15 & 0.14 \\
%%5371776 & Kepler-304 e & 1.49914 & 452.4 & 56.2 & 45.6 & 1.04 & 0.07 & 0.07 \\
%%5371776 & Kepler-304 d & 9.65349 & 37.9 & 4.2 & 4.0 & 2.29 & 0.13 & 0.13 \\
%%5371776 & Kepler-304 c & 5.31595 & 83.6 & 8.9 & 9.4 & 2.22 & 0.28 & 0.28 \\
%%5526527 & Kepler-970 b & 16.73648 & 10.8 & 1.3 & 1.2 & 2.74 & 0.76 & 0.62 \\
%%5531576 & Kepler-240 b & 4.14452 & 146.3 & 23.5 & 21.4 & 1.45 & 0.11 & 0.11 \\
%%5531576 & Kepler-240 c & 7.95355 & 60.9 & 8.6 & 8.6 & 2.19 & 0.13 & 0.13 \\
%%5617854 & Kepler-901 b & 3.51749 & 64.6 & 8.2 & 7.1 & 1.54 & 0.07 & 0.17 \\
%%5640085 & Kepler-159 b & 10.13959 & 28.2 & 3.2 & 2.9 & 2.22 & 0.11 & 0.10 \\
%%5640085 & Kepler-159 c & 43.58594 & 4.0 & 0.5 & 0.4 & 2.70 & 0.14 & 0.14 \\
%%5686174 & Kepler-622 b & 14.28228 & 11.5 & 1.3 & 1.2 & 2.09 & 0.11 & 0.09 \\
%%5688790 & Kepler-1439 b & 8.07394 & 8.6 & 0.9 & 0.8 & 1.77 & 0.09 & 0.39 \\
%%5706966 & Kepler-333 c & 24.08842 & 6.3 & 0.8 & 0.7 & 1.34 & 0.09 & 0.09 \\
%%5706966 & Kepler-333 b & 12.55111 & 15.0 & 1.9 & 1.7 & 1.55 & 0.39 & 0.09 \\
%%5794379 & Kepler-241 b & 12.71810 & 22.8 & 6.3 & 5.1 & 2.28 & 0.19 & 0.22 \\
%%5794379 & Kepler-241 c & 36.06588 & 5.7 & 1.4 & 1.2 & 2.63 & 0.22 & 0.24 \\
%%5868793 & Kepler-1582 b & 4.83809 & 4.3 & 0.4 & 0.4 & 0.85 & 0.07 & 0.06 \\
%%5940165 & Kepler-1062 b & 9.30414 & 30.0 & 6.3 & 5.0 & 1.85 & 0.14 & 0.19 \\
%%5980208 & Kepler-1331 b & 0.78916 & 523.6 & 120.1 & 87.0 & 0.90 & 0.07 & 0.08 \\
%%6020753 & Kepler-202 c & 16.28248 & 17.4 & 2.1 & 2.1 & 1.91 & 0.11 & 0.11 \\
%%6020753 & Kepler-202 b & 4.06943 & 111.5 & 12.6 & 12.5 & 1.55 & 0.09 & 0.07 \\
%%6026438 & Kepler-354 b & 5.47666 & 66.2 & 14.4 & 12.1 & 1.92 & 0.15 & 0.28 \\
%%6026438 & Kepler-354 d & 24.21020 & 9.2 & 2.0 & 1.6 & 1.43 & 0.15 & 0.15 \\
%%6026438 & Kepler-354 c & 16.93481 & 15.0 & 2.9 & 2.7 & 1.25 & 0.12 & 0.12 \\
%%6063220 & Kepler-99 b & 4.60358 & 113.0 & 12.8 & 13.1 & 1.51 & 0.07 & 0.07 \\
%%6186964 & Kepler-1366 b & 2.16457 & 119.0 & 12.3 & 11.5 & 1.57 & 0.07 & 0.09 \\
%%6205897 & Kepler-1029 b & 4.41769 & 94.1 & 10.5 & 9.0 & 1.52 & 0.25 & 0.37 \\
%%6263593 & Kepler-1418 b & 22.47660 & 7.1 & 1.4 & 1.2 & 1.21 & 0.15 & 0.11 \\
%%6382217 & Kepler-353 c & 8.41100 & 14.8 & 1.6 & 1.6 & 2.13 & 0.06 & 0.38 \\
%%6382217 & Kepler-353 b & 5.79529 & 24.4 & 2.7 & 2.4 & 1.25 & 0.35 & 0.35 \\
%%6425957 & Kepler-205 c & 20.30653 & 6.4 & 0.8 & 0.7 & 1.82 & 0.17 & 0.17 \\
%%6425957 & Kepler-205 b & 2.75564 & 93.1 & 11.1 & 10.0 & 1.58 & 0.09 & 0.12 \\
%%6435936 & Kepler-705 b & 56.05608 & 0.8 & 0.1 & 0.1 & 2.21 & 0.11 & 0.09 \\
%%6436029 & Kepler-1359 b & 59.49690 & 2.7 & 0.4 & 0.3 & 2.29 & 0.12 & 0.18 \\
%%6444896 & Kepler-1649 b & 8.68911 & 0.7 & nan & nan & 0.49 & 0.06 & 0.06 \\
%%6497146 & Kepler-438 b & 35.23307 & 1.7 & 0.3 & 0.3 & 0.97 & 0.09 & 0.08 \\
%%6665512 & Kepler-1048 b & 6.92101 & 33.0 & 8.4 & 7.3 & 2.46 & 0.34 & 0.30 \\
%%6697756 & Kepler-1351 b & 0.91614 & 412.9 & 80.8 & 83.5 & 0.55 & 0.05 & 0.04 \\
%%6773862 & Kepler-988 b & 17.76077 & 5.7 & 0.6 & 0.5 & 2.37 & 0.34 & 0.14 \\
%%6921944 & Kepler-1105 b & 4.42157 & 51.8 & 6.0 & 5.7 & 1.83 & 0.08 & 0.34 \\
%%6949607 & Kepler-28 c & 8.98582 & 32.9 & 4.1 & 3.7 & 1.88 & 0.10 & 0.10 \\
%%6949607 & Kepler-28 b & 5.91227 & 57.3 & 7.5 & 6.4 & 1.97 & 0.10 & 0.09 \\
%%6960913 & Kepler-61 b & 59.87803 & 1.5 & 0.1 & 0.1 & 2.48 & 0.11 & 0.08 \\
%%7021681 & Kepler-505 b & 27.52195 & 3.1 & 0.3 & 0.3 & 3.11 & 0.09 & 0.11 \\
%%7033233 & Kepler-1197 b & 2.03232 & 274.1 & 33.5 & 29.8 & 1.12 & 0.11 & 0.05 \\
%%7094486 & Kepler-1009 b & 11.35011 & 9.2 & 0.9 & 0.8 & 2.38 & 0.12 & 0.09 \\
%%7287995 & Kepler-81 d & 20.83763 & 7.3 & 0.8 & 0.8 & 1.44 & 0.36 & 0.36 \\
%%7287995 & Kepler-81 c & 12.03987 & 15.2 & 1.9 & 1.5 & 2.22 & 0.11 & 0.11 \\
%%7287995 & Kepler-81 b & 5.95490 & 38.6 & 4.3 & 4.1 & 2.37 & 0.11 & 0.12 \\
%%7304449 & Kepler-1650 b & 1.53818 & 43.1 & 4.5 & 4.3 & 1.34 & 0.08 & 0.26 \\
%%7350067 & Kepler-1646 b & 4.48559 & 9.1 & 1.5 & 1.3 & 1.84 & 0.09 & 0.14 \\
%%7439316 & Kepler-866 b & 2.61703 & 199.7 & 42.6 & 37.2 & 1.44 & 0.13 & 0.11 \\
%%7447200 & Kepler-210 b & 2.45324 & 102.7 & 10.4 & 9.2 & 3.00 & 0.18 & 0.18 \\
%%7455287 & Kepler-54 c & 12.07134 & 6.6 & 0.7 & 0.6 & 1.44 & 0.08 & 0.08 \\
%%7455287 & Kepler-54 d & 20.99588 & 3.1 & 0.3 & 0.3 & 1.46 & 0.07 & 0.07 \\
%%7455287 & Kepler-54 b & 8.01078 & 11.3 & 1.2 & 1.1 & 1.92 & 0.06 & 0.06 \\
%%7603200 & Kepler-138 b & 10.31282 & 9.5 & 1.1 & 0.9 & 0.63 & 0.03 & 0.03 \\
%%7603200 & Kepler-138 c & 13.78109 & 6.4 & 0.7 & 0.6 & 1.52 & 0.14 & 0.06 \\
%%7603200 & Kepler-138 d & 23.08902 & 3.2 & 0.3 & 0.3 & 1.32 & 0.05 & 0.05 \\
%%7676423 & Kepler-1430 b & 2.46050 & 203.6 & 55.6 & 47.1 & 1.04 & 0.12 & 0.11 \\
%%7802719 & Kepler-1157 b & 4.45743 & 81.1 & 17.3 & 14.0 & 0.93 & 0.09 & 0.07 \\
%%7826620 & Kepler-1579 b & 0.84991 & 376.6 & 110.2 & 86.1 & 0.56 & 0.06 & 0.07 \\
%%7870390 & Kepler-83 d & 5.16980 & 30.6 & 3.5 & 3.0 & 2.00 & 0.08 & 0.08 \\
%%7870390 & Kepler-83 b & 9.77045 & 13.1 & 1.4 & 1.3 & 2.69 & 0.10 & 0.08 \\
%%7870390 & Kepler-83 c & 20.09017 & 5.0 & 0.6 & 0.5 & 2.71 & 0.36 & 0.36 \\
%%7871954 & Kepler-303 c & 7.06117 & 22.4 & 2.3 & 2.3 & 1.49 & 0.18 & 0.18 \\
%%7871954 & Kepler-303 b & 1.93703 & 124.4 & 12.1 & 12.3 & 1.06 & 0.04 & 0.04 \\
%%7907423 & Kepler-249 d & 15.36846 & 3.9 & 0.4 & 0.4 & 1.38 & 0.06 & 0.06 \\
%%7907423 & Kepler-249 b & 3.30655 & 30.2 & 3.2 & 2.8 & 1.13 & 0.05 & 0.05 \\
%%7907423 & Kepler-249 c & 7.11373 & 10.9 & 1.2 & 1.1 & 1.45 & 0.05 & 0.06 \\
%%8009350 & Kepler-892 b & 13.75208 & 19.5 & 4.5 & 3.5 & 2.52 & 0.28 & 0.42 \\
%%8016691 & Kepler-1456 b & 18.13738 & 7.8 & 1.8 & 1.5 & 1.40 & 0.29 & 0.42 \\
%%8120608 & Kepler-186 e & 22.40778 & 3.1 & 0.3 & 0.3 & 1.40 & 0.06 & 0.06 \\
%%8120608 & Kepler-186 d & 13.34300 & 6.2 & 0.6 & 0.6 & 1.57 & 0.22 & 0.22 \\
%%8120608 & Kepler-186 c & 7.26731 & 13.8 & 1.4 & 1.3 & 1.39 & 0.05 & 0.04 \\
%%8120608 & Kepler-186 b & 3.88680 & 32.1 & 3.2 & 2.9 & 1.18 & 0.04 & 0.04 \\
%%8125580 & Kepler-749 b & 17.31712 & 19.0 & 3.7 & 3.5 & 3.21 & 0.70 & 0.30 \\
%%8150320 & Kepler-55 f & 10.19849 & 25.3 & 3.2 & 2.9 & 1.83 & 0.39 & 0.39 \\
%%8150320 & Kepler-55 e & 4.61749 & 72.9 & 9.1 & 8.4 & 1.70 & 0.22 & 0.22 \\
%%8150320 & Kepler-55 b & 27.95413 & 6.6 & 0.9 & 0.8 & 1.99 & 0.14 & 0.14 \\
%%8150320 & Kepler-55 c & 42.14061 & 3.8 & 0.5 & 0.4 & 1.94 & 0.21 & 0.18 \\
%%8150320 & Kepler-55 d & 2.21112 & 194.8 & 25.5 & 22.3 & 1.94 & 0.21 & 0.21 \\
%%8151055 & Kepler-1460 b & 29.96328 & 5.3 & 1.2 & 0.9 & 2.22 & 0.31 & 0.42 \\
%%8167996 & Kepler-327 c & 5.21230 & 24.5 & 2.6 & 2.3 & 1.24 & 0.05 & 0.05 \\
%%8167996 & Kepler-327 d & 13.96950 & 6.6 & 0.7 & 0.6 & 2.44 & 0.41 & 0.41 \\
%%8167996 & Kepler-327 b & 2.54956 & 64.0 & 6.5 & 5.8 & 1.33 & 0.06 & 0.04 \\
%%8183288 & Kepler-437 b & 66.65045 & 2.2 & 0.2 & 0.2 & 1.54 & 0.09 & 0.09 \\
%%8229458 & Kepler-1152 b & 1.64680 & 126.9 & 12.5 & 12.1 & 1.02 & 0.05 & 0.04 \\
%%8230616 & Kepler-316 b & 2.24049 & 148.0 & 18.0 & 17.8 & 1.45 & 0.06 & 0.10 \\
%%8230616 & Kepler-316 c & 6.82778 & 33.4 & 4.1 & 3.6 & 1.60 & 0.08 & 0.08 \\
%%8235924 & Kepler-1203 b & 0.58800 & 619.6 & 59.3 & 57.6 & 1.22 & 0.04 & 0.06 \\
%%8247771 & Kepler-1200 b & 1.11855 & 532.8 & 124.5 & 92.4 & 1.07 & 0.10 & 0.09 \\
%%8282651 & Kepler-1136 b & 2.36172 & 153.6 & 30.5 & 27.3 & 1.44 & 0.13 & 0.11 \\
%%8346392 & Kepler-777 b & 5.72813 & 30.6 & 3.0 & 3.0 & 1.62 & 0.08 & 0.08 \\
%%8351704 & Kepler-779 b & 7.09712 & 11.9 & 1.3 & 1.3 & 1.04 & 0.06 & 0.05 \\
%%8367644 & Kepler-993 b & 22.08552 & 3.9 & 0.4 & 0.4 & 3.48 & 0.58 & 0.30 \\
%%8424002 & Kepler-1512 b & 20.35972 & 1.5 & 0.7 & 0.6 & 0.79 & 0.12 & 0.17 \\
%%8463346 & Kepler-436 c & 16.79716 & 15.2 & 2.2 & 2.0 & 2.59 & 0.20 & 0.55 \\
%%8463346 & Kepler-436 b & 64.00197 & 2.6 & 0.4 & 0.3 & 2.34 & 0.51 & 0.51 \\
%%8505670 & Kepler-252 c & 10.84845 & 18.6 & 2.1 & 2.2 & 2.76 & 0.14 & 0.12 \\
%%8505670 & Kepler-252 b & 6.66833 & 35.8 & 4.1 & 3.8 & 1.39 & 0.54 & 0.54 \\
%%8544992 & Kepler-388 c & 13.29707 & 16.5 & 3.9 & 3.3 & 0.91 & 0.10 & 0.09 \\
%%8544992 & Kepler-388 b & 3.17323 & 110.0 & 25.0 & 21.8 & 0.89 & 0.09 & 0.09 \\
%%8547140 & Kepler-801 b & 11.41926 & 11.5 & 1.2 & 1.2 & 2.07 & 0.07 & 0.07 \\
%%8561063 & Kepler-42 b & 1.21377 & 22.6 & 2.5 & 2.3 & 0.91 & 0.03 & 0.12 \\
%%8561063 & Kepler-42 d & 1.86511 & 12.8 & 1.3 & 1.3 & 0.77 & 0.09 & 0.09 \\
%%8733898 & Kepler-446 d & 5.14892 & 4.9 & 0.5 & 0.4 & 1.14 & 0.23 & 0.23 \\
%%8733898 & Kepler-446 c & 3.03621 & 9.8 & 1.1 & 1.0 & 0.93 & 0.05 & 0.05 \\
%%8733898 & Kepler-446 b & 1.56541 & 23.6 & 2.4 & 2.3 & 1.15 & 0.06 & 0.04 \\
%%8826007 & Kepler-1450 b & 54.50915 & 2.7 & 0.7 & 0.5 & 2.33 & 0.21 & 0.60 \\
%%8845205 & Kepler-560 b & 18.47763 & 1.8 & 0.2 & 0.2 & 2.21 & 0.18 & 0.15 \\
%%8874090 & Kepler-834 b & 13.32389 & 9.0 & 1.0 & 0.9 & 1.97 & 0.10 & 0.08 \\
%%8890150 & Kepler-395 c & 34.98978 & 2.2 & 0.2 & 0.2 & 1.41 & 0.11 & 0.08 \\
%%8890150 & Kepler-395 b & 7.05429 & 18.8 & 2.0 & 1.7 & 1.24 & 0.11 & 0.11 \\
%%8978528 & Kepler-329 b & 7.41635 & 32.3 & 4.0 & 3.6 & 1.72 & 0.12 & 0.12 \\
%%8978528 & Kepler-329 c & 18.68484 & 9.4 & 1.2 & 1.0 & 2.68 & 0.13 & 0.16 \\
%%9112931 & Kepler-1161 b & 10.71253 & 21.3 & 2.6 & 2.6 & 2.86 & 0.35 & 0.27 \\
%%9214942 & Kepler-833 b & 18.75470 & 7.0 & 0.7 & 0.6 & 2.14 & 0.11 & 0.08 \\
%%9334893 & Kepler-1178 b & 31.80551 & 4.8 & 1.0 & 0.9 & 1.53 & 0.12 & 0.13 \\
%%9351316 & Kepler-1086 b & 18.78429 & 8.7 & 1.1 & 1.0 & 2.24 & 0.50 & 0.21 \\
%%9353314 & Kepler-1007 b & 5.18499 & 40.1 & 8.3 & 7.7 & 1.14 & 0.11 & 0.12 \\
%%9388479 & Kepler-732 b & 9.46782 & 7.9 & 0.8 & 0.8 & 2.45 & 0.08 & 0.07 \\
%%9388479 & Kepler-732 c & 0.89304 & 182.7 & 19.6 & 18.4 & 1.42 & 0.05 & 0.05 \\
%%9390653 & Kepler-504 b & 9.54928 & 5.5 & 0.6 & 0.6 & 1.77 & 0.06 & 0.05 \\
%%9411412 & Kepler-1096 b & 2.89222 & 111.7 & 12.7 & 12.5 & 1.51 & 0.28 & 0.21 \\
%%9412760 & Kepler-345 c & 9.38741 & 26.5 & 2.9 & 2.8 & 1.71 & 0.11 & 0.08 \\
%%9412760 & Kepler-345 b & 7.41556 & 36.1 & 4.0 & 3.2 & 0.79 & 0.07 & 0.06 \\
%%9447166 & Kepler-1459 b & 62.86946 & 3.4 & 0.4 & 0.4 & 1.95 & 0.08 & 0.53 \\
%%9475552 & Kepler-1315 b & 0.84338 & 949.2 & 121.9 & 112.4 & 1.41 & 0.23 & 0.13 \\
%%9573685 & Kepler-1074 b & 5.94566 & 26.1 & 2.9 & 2.6 & 1.18 & 0.06 & 0.04 \\
%%9631762 & Kepler-615 b & 10.35586 & 34.1 & 5.3 & 4.6 & 1.92 & 0.09 & 0.11 \\
%%9710326 & Kepler-737 b & 28.59914 & 2.1 & 0.2 & 0.2 & 2.03 & 0.07 & 0.06 \\
%%9718066 & Kepler-378 b & 16.09228 & 17.3 & 2.2 & 1.8 & 1.05 & 0.04 & 0.26 \\
%%9718066 & Kepler-378 c & 28.90605 & 7.9 & 1.0 & 0.8 & 0.84 & 0.18 & 0.18 \\
%%9730163 & Kepler-445 c & 4.87122 & 5.8 & 0.6 & 0.6 & 2.65 & 0.30 & 0.25 \\
%%9730163 & Kepler-445 b & 2.98416 & 11.2 & 1.2 & 1.1 & 1.56 & 0.11 & 0.11 \\
%%9757613 & Kepler-26 e & 46.82768 & 1.6 & 0.2 & 0.1 & 2.40 & 0.10 & 0.10 \\
%%9757613 & Kepler-26 b & 12.28300 & 9.5 & 1.0 & 0.9 & 3.19 & 0.10 & 0.10 \\
%%9757613 & Kepler-26 d & 3.54392 & 49.3 & 4.6 & 4.6 & 1.36 & 0.05 & 0.05 \\
%%9757613 & Kepler-26 c & 17.25121 & 6.0 & 0.7 & 0.6 & 2.98 & 0.28 & 0.23 \\
%%9787239 & Kepler-32 f & 0.74296 & 314.4 & 32.0 & 28.0 & 1.70 & 0.42 & 0.42 \\
%%9787239 & Kepler-32 e & 2.89601 & 50.9 & 5.4 & 5.2 & 1.42 & 0.06 & 0.06 \\
%%9787239 & Kepler-32 b & 5.90128 & 19.7 & 2.0 & 2.0 & 2.23 & 0.07 & 0.07 \\
%%9787239 & Kepler-32 c & 8.75210 & 11.7 & 1.3 & 1.1 & 2.11 & 0.08 & 0.08 \\
%%9787239 & Kepler-32 d & 22.78078 & 3.3 & 0.3 & 0.3 & 2.50 & 0.09 & 0.09 \\
%%9823519 & Kepler-1350 b & 4.49686 & 24.5 & 4.0 & 3.7 & 2.29 & 0.10 & 0.09 \\
%%9823519 & Kepler-1350 c & 1.76679 & 84.9 & 12.8 & 11.9 & 1.66 & 0.08 & 0.08 \\
%%9837661 & Kepler-1321 c & 2.22650 & 91.2 & 11.2 & 10.1 & 3.03 & 0.18 & 0.18 \\
%%9875711 & Kepler-742 b & 8.36087 & 41.4 & 8.6 & 7.5 & 3.10 & 0.38 & 0.24 \\
%%9881077 & Kepler-1337 b & 24.40044 & 8.7 & 2.1 & 1.6 & 2.54 & 0.20 & 0.58 \\
%%9941066 & Kepler-898 b & 5.87063 & 35.3 & 4.7 & 4.1 & 1.50 & 0.30 & 0.12 \\
%%9950612 & Kepler-220 b & 4.15982 & 98.9 & 11.5 & 10.1 & 0.92 & 0.20 & 0.20 \\
%%9950612 & Kepler-220 c & 9.03419 & 35.1 & 4.5 & 3.7 & 1.93 & 0.08 & 0.30 \\
%%9950612 & Kepler-220 d & 28.12244 & 7.7 & 0.9 & 0.8 & 0.96 & 0.25 & 0.25 \\
%%9950612 & Kepler-220 e & 45.90298 & 4.0 & 0.4 & 0.5 & 1.27 & 0.06 & 0.06 \\
%%10005788 & Kepler-1022 b & 10.99471 & 17.7 & 3.9 & 3.3 & 1.86 & 0.15 & 0.15 \\
%%10027247 & Kepler-1229 b & 86.82952 & 0.5 & 0.0 & 0.0 & 2.06 & 0.06 & 0.56 \\
%%10027323 & Kepler-309 b & 5.92366 & 77.1 & 9.3 & 9.0 & 1.50 & 0.09 & 0.07 \\
%%10059645 & Kepler-1265 b & 6.49441 & 57.8 & 13.3 & 11.0 & 1.62 & 0.16 & 0.38 \\
%%10122538 & Kepler-1388 c & 5.53609 & 69.9 & 8.9 & 8.1 & 1.90 & 0.35 & 0.35 \\
%%10122538 & Kepler-1388 d & 20.95700 & 11.9 & 1.4 & 1.4 & 2.05 & 0.13 & 0.13 \\
%%10122538 & Kepler-1388 b & 12.28547 & 24.0 & 3.2 & 2.8 & 1.90 & 0.12 & 0.12 \\
%%10122538 & Kepler-1388 e & 37.63339 & 5.4 & 0.7 & 0.6 & 1.94 & 0.13 & 0.13 \\
%%10166274 & Kepler-267 b & 3.35373 & 40.9 & 4.0 & 3.8 & 1.92 & 0.07 & 0.06 \\
%%10166274 & Kepler-267 c & 6.87745 & 15.6 & 1.6 & 1.3 & 2.39 & 0.29 & 0.29 \\
%%10166274 & Kepler-267 d & 28.46465 & 2.3 & 0.2 & 0.2 & 2.18 & 0.10 & 0.10 \\
%%10190777 & Kepler-1019 b & 1.41123 & 348.7 & 37.6 & 35.8 & 1.49 & 0.06 & 0.23 \\
%%10318874 & Kepler-94 b & 2.50806 & 229.9 & 28.2 & 21.8 & 3.19 & 0.14 & 0.25 \\
%%10328458 & Kepler-1146 b & 2.35227 & 245.2 & 54.6 & 42.3 & 1.22 & 0.12 & 0.10 \\
%%10329835 & Kepler-1075 b & 1.52373 & 128.3 & 14.3 & 11.1 & 1.47 & 0.11 & 0.35 \\
%%10332883 & Kepler-994 b & 1.15117 & 227.8 & 24.2 & 22.3 & 1.41 & 0.05 & 0.04 \\
%%10340423 & Kepler-225 b & 6.73894 & 34.2 & 4.6 & 4.1 & 1.73 & 0.12 & 0.12 \\
%%10340423 & Kepler-225 c & 18.79419 & 8.7 & 1.1 & 1.0 & 2.81 & 0.15 & 0.16 \\
%%10386984 & Kepler-658 b & 1.28708 & 188.6 & 18.8 & 18.8 & 1.78 & 0.08 & 0.06 \\
%%10388286 & Kepler-617 b & 1.68270 & 91.4 & 9.5 & 8.9 & 1.42 & 0.06 & 0.04 \\
%%10489206 & Kepler-125 c & 5.77440 & 20.6 & 1.9 & 2.0 & 0.86 & 0.06 & 0.06 \\
%%10489206 & Kepler-125 b & 4.16438 & 31.9 & 3.4 & 2.6 & 2.73 & 0.08 & 0.14 \\
%%10525027 & Kepler-1049 b & 3.27345 & 45.9 & 4.4 & 4.3 & 0.96 & 0.03 & 0.05 \\
%%10583066 & Kepler-661 b & 6.02930 & 48.8 & 9.8 & 9.5 & 2.71 & 0.22 & 0.21 \\
%%10591855 & Kepler-1367 b & 1.57409 & 237.2 & 29.2 & 27.9 & 0.90 & 0.08 & 0.05 \\
%%10599206 & Kepler-566 b & 18.42795 & 16.8 & 1.9 & 1.9 & 2.16 & 0.36 & 0.22 \\
%%10604335 & Kepler-283 b & 11.00818 & 15.1 & 1.9 & 1.8 & 2.37 & 0.14 & 0.11 \\
%%10604335 & Kepler-283 c & 92.74958 & 0.9 & 0.1 & 0.1 & 2.05 & 0.15 & 0.15 \\
%%10670119 & Kepler-369 b & 2.73277 & 53.5 & 9.4 & 7.8 & 1.44 & 0.22 & 0.19 \\
%%10670119 & Kepler-369 c & 14.87150 & 5.5 & 0.9 & 0.9 & 1.60 & 0.14 & 0.12 \\
%%10717241 & Kepler-551 b & 12.37647 & 13.4 & 1.6 & 1.4 & 2.67 & 0.31 & 0.13 \\
%%10842192 & Kepler-1329 b & 9.33648 & 25.7 & 6.0 & 4.5 & 2.40 & 0.19 & 0.52 \\
%%10905746 & Kepler-1651 b & 9.87865 & 5.9 & 2.0 & 1.5 & 1.60 & 0.21 & 0.21 \\
%%10925104 & Kepler-114 b & 5.18856 & 78.3 & 10.0 & 8.6 & 1.33 & 0.06 & 0.06 \\
%%10925104 & Kepler-114 d & 11.77613 & 26.3 & 3.2 & 2.7 & 2.76 & 0.22 & 0.22 \\
%%10925104 & Kepler-114 c & 8.04135 & 43.6 & 5.0 & 4.9 & 1.72 & 0.08 & 0.08 \\
%%10975146 & Kepler-808 b & 0.63133 & 716.3 & 170.8 & 144.4 & 1.18 & 0.15 & 0.14 \\
%%10990886 & Kepler-568 b & 11.02347 & 8.2 & 0.8 & 0.7 & 2.51 & 0.29 & 0.23 \\
%%11129738 & Kepler-844 b & 2.61302 & 76.6 & 8.5 & 8.3 & 1.81 & 0.06 & 0.28 \\
%%11176127 & Kepler-298 c & 22.92886 & 8.2 & 1.0 & 0.9 & 2.30 & 0.45 & 0.45 \\
%%11176127 & Kepler-298 d & 77.47410 & 1.6 & 0.2 & 0.2 & 2.90 & 0.46 & 0.46 \\
%%11176127 & Kepler-298 b & 10.47548 & 23.1 & 3.0 & 2.7 & 2.33 & 0.39 & 0.19 \\
%%11194032 & Kepler-533 b & 28.51118 & 8.7 & 1.0 & 1.0 & 3.37 & 0.15 & 0.36 \\
%%11236244 & Kepler-1246 b & 11.32270 & 24.8 & 5.4 & 4.0 & 1.64 & 0.13 & 0.46 \\
%%11348997 & Kepler-1089 b & 5.13249 & 26.8 & 2.6 & 2.8 & 1.64 & 0.08 & 0.07 \\
%%11453592 & Kepler-1319 b & 2.88676 & 32.8 & 3.3 & 3.0 & 1.60 & 0.07 & 0.34 \\
%%11495458 & Kepler-1190 b & 10.45840 & 25.5 & 3.0 & 2.9 & 1.09 & 0.09 & 0.06 \\
%%11497958 & Kepler-296 f & 63.33547 & 0.4 & 0.2 & 0.1 & 1.18 & 0.17 & 0.19 \\
%%11497958 & Kepler-296 e & 34.14205 & 0.9 & 0.4 & 0.3 & 1.05 & 0.17 & 0.17 \\
%%11497958 & Kepler-296 d & 19.85029 & 1.8 & 0.7 & 0.6 & 1.53 & 0.21 & 0.25 \\
%%11497958 & Kepler-296 c & 5.84164 & 9.0 & 3.2 & 2.7 & 1.39 & 0.19 & 0.21 \\
%%11497958 & Kepler-296 b & 10.86441 & 4.1 & 1.6 & 1.2 & 1.19 & 0.32 & 0.24 \\
%%11611600 & Kepler-842 b & 1.21957 & 612.5 & 126.1 & 101.8 & 1.52 & 0.13 & 0.11 \\
%%11622600 & Kepler-991 b & 82.53412 & 1.3 & 0.1 & 0.1 & 2.74 & 0.15 & 0.12 \\
%%11754553 & Kepler-52 d & 36.44540 & 2.9 & 0.4 & 0.3 & 2.01 & 0.13 & 0.13 \\
%%11754553 & Kepler-52 c & 16.38490 & 8.5 & 1.1 & 0.9 & 2.02 & 0.12 & 0.09 \\
%%11754553 & Kepler-52 b & 7.87742 & 22.4 & 2.9 & 2.5 & 2.18 & 0.13 & 0.13 \\
%%11768142 & Kepler-1652 b & 38.09707 & 0.8 & 0.2 & 0.2 & 1.50 & 0.27 & 0.32 \\
%%11853255 & Kepler-674 b & 2.24338 & 132.7 & 16.6 & 15.8 & 2.31 & 0.11 & 0.40 \\
%%11853878 & Kepler-968 b & 3.69299 & 87.6 & 18.4 & 16.6 & 2.08 & 0.37 & 0.25 \\
%%11853878 & Kepler-968 c & 5.70941 & 49.8 & 10.7 & 8.3 & 1.81 & 0.44 & 0.44 \\
%%11923270 & Kepler-676 b & 11.59822 & 7.5 & 0.8 & 0.7 & 3.14 & 0.33 & 0.22 \\
%%12066335 & Kepler-231 b & 10.06524 & 14.3 & 1.5 & 1.3 & 2.00 & 0.30 & 0.30 \\
%%12066335 & Kepler-231 c & 19.27154 & 6.0 & 0.6 & 0.6 & 2.03 & 0.09 & 0.07 \\
%%12066569 & Kepler-1455 b & 49.27684 & 1.4 & 0.1 & 0.1 & 1.91 & 0.11 & 0.09 \\
%%12252424 & Kepler-113 c & 8.92508 & 44.1 & 5.5 & 4.8 & 2.54 & 0.11 & 0.25 \\
%%12252424 & Kepler-113 b & 4.75400 & 101.3 & 11.3 & 10.3 & 2.16 & 0.19 & 0.19 \\
%%12302530 & Kepler-155 b & 5.93120 & 23.1 & 4.1 & 3.6 & 1.94 & 0.17 & 0.20 \\
%%12302530 & Kepler-155 c & 52.66153 & 1.3 & 0.2 & 0.2 & 1.92 & 0.26 & 0.22 \\
%%12506770 & Kepler-895 b & 2.80624 & 97.3 & 11.4 & 11.4 & 1.58 & 0.10 & 0.09 
\enddata
\tablecomments{Only the first five rows are shown here to illustrate the table's content and format. The complete table in csv format is available in the arXiv source.}
\end{deluxetable*}
\capstarttrue

\capstartfalse
\begin{deluxetable*}{ccccccccc}
\tabletypesize{\small}
\tablecaption{K2 confirmed planet parameters\label{table:planetsK2}}
\tablehead{EPIC & Planet & $P$ & $F$ & $F$ upper limit & $F$ lower limit & $r_p$ & $r_p$ upper limit & $r_p$ lower limit \\
& name & [days] & [F$_{\oplus}$] & [F$_{\oplus}$] & [F$_{\oplus}$] & [R$_{\oplus}$] & [R$_{\oplus}$] & [R$_{\oplus}$]}
\startdata
201110617 & K2-156 b & 0.81315 & 615.4 & 51.0 & 55.4 & 1.35 & 0.12 & 0.10 \\
201155177 & K2-42 b & 6.68796 & 54.8 & 6.7 & 5.7 & 2.45 & 0.27 & 0.25 \\
201205469 & K2-43 c & 2.19888 & 81.8 & 8.5 & 7.9 & 1.43 & 0.09 & 0.08 \\
201205469 & K2-43 b & 3.47114 & 44.4 & 4.9 & 4.3 & 2.66 & 0.17 & 0.13 \\
201208431 & K2-4 b & 10.00440 & 16.5 & 1.8 & 1.6 & 2.52 & 0.34 & 0.31 
%%201338508 & K2-5 b & 5.73597 & 29.8 & 2.5 & 2.6 & 1.95 & 0.17 & 0.18 \\
%%201338508 & K2-5 c & 10.93240 & 12.6 & 1.2 & 1.1 & 2.15 & 0.20 & 0.20 \\
%%201367065 & K2-3 d & 44.56090 & 1.4 & 0.2 & 0.1 & 1.42 & 0.11 & 0.10 \\
%%201367065 & K2-3 c & 24.64350 & 3.1 & 0.3 & 0.3 & 1.73 & 0.12 & 0.11 \\
%%201367065 & K2-3 b & 10.05443 & 10.2 & 1.1 & 1.0 & 2.11 & 0.12 & 0.12 \\
%%201549860 & K2-35 c & 5.60906 & 56.4 & 12.7 & 10.6 & 1.99 & 0.25 & 0.24 \\
%%201549860 & K2-35 b & 2.39991 & 174.5 & 36.7 & 29.7 & 1.27 & 0.17 & 0.18 \\
%%201635569 & K2-14 b & 8.36926 & 8.6 & 2.1 & 1.9 & 4.51 & 0.60 & 0.57 \\
%%201690311 & K2-49 b & 2.77065 & 125.8 & 20.5 & 17.5 & 3.20 & 0.34 & 0.33 \\
%%201833600 & K2-50 b & 8.75290 & 25.9 & 5.4 & 4.4 & 1.93 & 0.29 & 0.28 \\
%%201855371 & K2-17 b & 17.96540 & 7.7 & 0.6 & 0.5 & 2.04 & 0.20 & 0.19 \\
%%201912552 & K2-18 b & 32.94180 & 1.1 & 0.1 & 0.1 & 2.54 & 0.14 & 0.14 \\
%%202083828 & K2-26 b & 14.56620 & 5.2 & 0.8 & 0.7 & 2.55 & 0.20 & 0.21 \\
%%205916793 & K2-54 b & 9.78430 & 13.9 & 2.0 & 1.8 & 1.90 & 0.21 & 0.20 \\
%%205924614 & K2-55 b & 2.84926 & 134.2 & 30.2 & 26.6 & 4.16 & 0.39 & 0.37 \\
%%206011691 & K2-21 b & 9.32389 & 16.8 & 1.1 & 1.2 & 1.74 & 0.11 & 0.12 \\
%%206011691 & K2-21 c & 15.50116 & 8.5 & 0.6 & 0.6 & 2.13 & 0.15 & 0.14 \\
%%206026136 & K2-57 b & 9.00630 & 27.4 & 2.7 & 2.2 & 2.25 & 0.23 & 0.21 \\
%%206096602 & K2-62 c & 16.19660 & 15.2 & 3.3 & 2.8 & 2.00 & 0.20 & 0.21 \\
%%206096602 & K2-62 b & 6.67177 & 49.6 & 10.8 & 9.1 & 2.04 & 0.19 & 0.20 \\
%%206159027 & K2-68 b & 8.05428 & 43.4 & 6.8 & 6.0 & 1.58 & 0.16 & 0.14 \\
%%206162305 & K2-69 b & 7.06599 & 20.5 & 3.9 & 3.5 & 2.47 & 0.23 & 0.19 \\
%%206192813 & K2-71 b & 6.98541 & 16.1 & 2.4 & 2.4 & 2.37 & 0.25 & 0.23 \\
%%206209135 & K2-72 d & 7.75990 & 5.2 & 1.0 & 0.8 & 1.02 & 0.14 & 0.11 \\
%%206209135 & K2-72 c & 15.18710 & 2.1 & 0.4 & 0.3 & 1.20 & 0.13 & 0.12 \\
%%206209135 & K2-72 e & 24.16690 & 1.1 & 0.2 & 0.2 & 1.14 & 0.14 & 0.13 \\
%%206209135 & K2-72 b & 5.57739 & 7.9 & 1.4 & 1.4 & 1.06 & 0.11 & 0.10 \\
%%210448987 & K2-81 b & 6.10224 & 65.2 & 18.4 & 14.9 & 2.25 & 0.29 & 0.25 \\
%%210508766 & K2-83 b & 2.74697 & 64.2 & 6.2 & 5.1 & 1.65 & 0.13 & 0.13 \\
%%210508766 & K2-83 c & 9.99767 & 11.5 & 1.0 & 1.0 & 1.97 & 0.13 & 0.13 \\
%%210558622 & K2-174 b & 19.56400 & 10.9 & 2.5 & 1.9 & 2.53 & 0.27 & 0.24 \\
%%210707130 & K2-85 b & 0.68455 & 869.3 & 166.9 & 153.0 & 1.49 & 0.17 & 0.15 \\
%%210750726 & K2-88 b & 4.61220 & 15.8 & 2.1 & 1.8 & 2.10 & 0.14 & 0.13 \\
%%210838726 & K2-89 b & 1.09603 & 161.5 & 14.3 & 14.9 & 1.02 & 0.09 & 0.10 \\
%%210968143 & K2-90 b & 13.73110 & 11.6 & 2.2 & 1.8 & 2.71 & 0.23 & 0.24 \\
%%210968143 & K2-90 c & 2.90032 & 93.9 & 16.9 & 15.4 & 1.37 & 0.17 & 0.16 \\
%%211077024 & K2-91 b & 1.41955 & 67.0 & 8.4 & 7.0 & 1.45 & 0.12 & 0.11 \\
%%211969807 & K2-104 b & 1.97499 & 69.4 & 13.2 & 11.3 & 2.04 & 0.13 & 0.13 \\
%%212069861 & K2-123 b & 30.95422 & 2.7 & 0.3 & 0.3 & 2.87 & 0.11 & 0.11 \\
%%212154564 & K2-124 b & 6.41401 & 10.0 & 1.5 & 1.3 & 3.15 & 0.13 & 0.13 \\
%%212460519 & K2-126 b & 7.38707 & 29.2 & 1.8 & 1.7 & 1.83 & 0.14 & 0.10 \\
%%212686205 & K2-128 b & 5.67581 & 58.0 & 4.6 & 4.4 & 1.20 & 0.10 & 0.08 \\
%%212779596 & K2-199 c & 7.37450 & 44.0 & 5.3 & 3.9 & 2.83 & 0.18 & 0.17 \\
%%212779596 & K2-199 b & 3.22542 & 133.7 & 15.2 & 13.7 & 1.88 & 0.16 & 0.13 \\
%%217941732 & K2-130 b & 2.49416 & 140.7 & 15.1 & 12.8 & 1.16 & 0.12 & 0.12 \\
%%220241529 & K2-209 b & 2.08061 & 290.0 & 59.1 & 56.0 & 0.92 & 0.13 & 0.10 \\
%%220321605 & K2-212 b & 9.79540 & 17.3 & 1.6 & 1.6 & 2.72 & 0.16 & 0.13 \\
%%220481411 & K2-216 b & 2.17479 & 209.6 & 17.2 & 15.5 & 1.69 & 0.10 & 0.08 
\enddata
\tablecomments{Only the first five rows are shown here to illustrate the table's content and format.}
\end{deluxetable*}
\capstarttrue



\section{Transiting Planet Detection Completeness}  \label{sect:completeness}
Derivation of the planet occurrence rate requires the empirical distribution of planet detections to be corrected
for imperfect survey completeness. The completeness correction is treated separately for each subset
of planets from \kepler{} or \ktwo{} in the following subsections. Each set of corrections accounts for detection
biases arising from the imperfect detection sensitivity and for the 
geometric probability of a planetary transit to occur.

\subsection{Kepler Sensitivity}
The derivation of the \kepler{} planet detection sensitivity follows from the methodology outlined in
\cite{christiansen16} and used by \cite{fulton17} to resolve the radius valley around FGK stars. Per-target
\kepler{} completeness products for DR25 and the SOC 9.3 version of the \kepler{} pipeline
\citep{jenkins10} are available
for all of the planet-host stars in our \kepler{} sample \citep{burke15,burke17}. Detection sensitivities
(or efficiencies) were calculated via transiting planetary signal injections at the pixel level
\citep{christiansen15,christiansen17} which are subsequently processed by the \kepler{} pipeline
transiting planet search (TPS) module from
which the detection sensitivity as a function of the Multi-event statistic (MES) is computed as the fraction of
injected signals that are successfully recovered by the pipeline.

The MES represents the level of significance of a transit signal at a specified transit duration ranging from
1.5-15 hours. Following \cite{petigura18} we adopt an alternative diagnostic for the transit signal significance
in the form of the transit S/N 

\begin{equation}
  \text{S/N} = \frac{Z}{\text{CDPP}_{D}} \sqrt{n_{\text{transits}}(\mathbf{t},P,T_0)}  \label{eq:snr}
\end{equation}

\noindent where $Z=(r_p/R_s)^2$ is the transit depth assuming a non-grazing transit (i.e. $b\lesssim 0.9$),
CDPP$_D$ is the Combined Differential Photometric
Precision on the timescale of the transit duration $D$, and $n_{\text{transits}}$ is the number of
observed transits given the target's data span and duty cycle\footnote{Number of useful cadences for transit signal
detection.} of observations $\textbf{t}$, the planet's orbital period $P$, and its time of mid-transit $T_0$.

To compute the \kepler{} detection sensitivity as a function of S/N, we first
derive the mapping between the MES and the transit S/N using the data from \cite{christiansen15} who
derived the detection sensitivity of the \kepler{} pipeline from one year of data. 
The parameters of the injected planets are provided along with their corresponding MES and CDPP at each value of
$D$ considered. For each injected planet we interpolate its MES and CDPP values to $D$ and calculate
the transit S/N using Eq.~\ref{eq:snr}. The mapping between MES and S/N is shown in Fig.~\ref{fig:messnr}
for the full set of injected planets whose transit S/N values span 2.7-4843. Given the large number of injected planetary
signals ($>10^4$), we fit the number-weighted S/N to MES mapping using the \texttt{scipy.curve\_fit} non-linear least
squares algorithm with a powerlaw function of the form $\text{MES} = A\cdot \text{S/N}^{\alpha}$. We find a best-fit
amplitude and powerlaw index of $A=0.977$ and $\alpha=0.967$ respectively with negligible uncertainties.
This relation is used to map the transit S/N to MES
which is then mapped to the detection sensitivity. The average \kepler{} detection sensitivity curve as a function of
transit S/N, along with the $16^{\text{th}}$ and $84^{\text{th}}$ percentiles for the stars in our \kepler{} sample
are shown in Fig.~\ref{fig:senscurves}.


\begin{figure}
  \centering
  %\includegraphics[width=0.98\hsize]{figures/MES_SNR.png}
  \caption{Correlation between the \kepler{} multi-event statistic and transit S/N. The mapping between the MES and
    S/N based on the synthetic planetary signals injected into the \kepler{} pipeline \citep{christiansen15}.
    The number-weighted powerlaw fit (\emph{solid line}) to the correlation differs slightly from a one-to-one relation
    (\emph{dashed line}) with the.}  
  \label{fig:messnr}
\end{figure}


\begin{figure}
  \centering
  %\includegraphics[width=0.98\hsize]{figures/senscurves.png}
  \caption{Average detection sensitivity for \kepler{} and \ktwo{.} The \emph{solid curves} represent the
    average transiting planet detection sensitivity for the \kepler{} and \ktwo{} stars in our sample as
    a function of the transit S/N (Eq.~\ref{eq:snr}). The shaded regions mark the 16$^{th}$ and 84$^{th}$
    percentiles of the measured detection sensitivities.} 
  \label{fig:senscurves}
\end{figure}


\subsection{K2 Sensitivity}
Unlike the primary \kepler{} mission, the \ktwo{} data products do not feature detailed completeness and reliability
products. To derive the detection sensitivity among the \ktwo{} stars in our sample we employ the transit detection
pipeline \texttt{ORION} \citep{cloutier19b}.
%\texttt{ORION} was initially built to search for planets in the 2-minute
%cadence light curves from \tess{.} Because of the longer 30-minute cadence of \ktwo{} light curves
The failure of the second reaction wheel on board the \kepler{} spacecraft in 2013 prevented the observatory from
maintaining the fine pointing accuracy required to obtain ultra precise photometry. The re-purposed \ktwo{}
mission exploited the solar wind pressure by enabling the observatory to continue pointing along the ecliptic
plane with realignments via thrusters firings \textbf{every $\sim 6$ hours} \citep{howell14}.
\texttt{ORION} lacks a module to correct for the temporally correlated pointing corrections so as input we feed
the \texttt{EVEREST} light curves which use a pixel level decorrelation to remove systematics from the spacecraft's
variable pointing \citep{luger16,luger18}. We favor the \texttt{EVEREST} \ktwo{} light curves over light curves
produced by similar pipelines (e.g. \texttt{K2SFF}; \citealt{vanderburg14}, \texttt{K2SC};
\citealt{aigrain15,aigrain16})
due to its performance in obtaining improved photometric precision at the level of $\sim 20-50$\% \citep{luger16}.

We quantify the \ktwo{} detection sensitivity using \texttt{ORION} by first retrieving the \texttt{EVEREST} light
curve from MAST for each star in our sample. We only consider light curves from individual campaigns. As
\texttt{ORION} input we supply the time sampling $\mathbf{t}$ in BJD, the corrected flux, and flux uncertainties
in e$^-$/second, from the \texttt{EVEREST} keywords \texttt{TIME}, \texttt{FCOR}, and \texttt{FRAW\_ERR}.
The duty cycle is derived by restricting ourselves to light curve
measurements for which the \texttt{QUALITY} flag is zero.
Any known signals from planets, PCs, or FPs are removed from the light
curve based on their reported transit parameters using the \texttt{batman} \citep{kreidberg15} implementation of
the \cite{mandel02} transit model. We then inject transiting planetary signals directly into the light curve
by sampling planets from the linear transit S/N grid $\mathcal{U}(0,50)$. The per-system
multiplicity is drawn
from the cumulative occurrence rate of small planets out to 200 days around
mid-K to early M dwarfs from \kepler{} \citep[$2.5 \pm 0.2$;][]{dressing15a}.
Each planet's time of mid-transit $T_0$ is
drawn from $\mathcal{U}(\text{min}(\mathbf{t}),\text{max}(\mathbf{t}))$.
In a given light curve, with a fixed CDPP($D$) and $\mathbf{t}$, for a star
whose radius and mass are fixed to their maximum likelihood values, we
draw each planet's logarithmic orbital period from $\mathcal{U}(\log_{10}(0.5\text{ days}),\log_{10}(200\text{ days}))$
which allows us to compute the number of
transits that occur within $\mathbf{t}$. Note that some injected planets
will exhibit $n_{\text{transits}}=0$ due to the limited \ktwo{} baselines of typically $\sim 80$ days.
The drawn orbital period also uniquely determines the planet's radius corresponding to its drawn value of the S/N.
To ensure dynamical stability in multi-planet
systems, we compute the maximum likelihood planet mass from the probabilistic mass-radius relation \texttt{forecaster}
\citep{chen17} and analytically assess the Lagrange stability of each neighboring planet pair assuming circular
orbits \citep{barnes06}. Each planet's scaled semimajor axis $a/R_s$ and scaled radius $r_p/R_s$ follow from
their sampled radius $r_p$ and the stellar parameters $R_s$ and $M_s$. We sample impact parameters $b$ from
$\mathcal{U}(0,0.9)$ to compute the orbital inclinations. Furthermore, we adopt fixed quadratic limb darkening
coefficients by interpolating the \kepler{} bandpass coefficient grid along \teff{,} \logg{,} and [Fe/H], assuming
solar metallicity when [Fe/H] measurements are absent \citep{claret12}.
These parameters are used to compute transit models in the absence of any transit timing variations.
Transit signals are then injected into the cleaned \ktwo{} light curves and fed to
\texttt{ORION} to conduct a blind search for transiting signals.

The detection sensitivity as a function of S/N for each \ktwo{} star is computed by considering a minimum of
$10^2$ injected planetary systems per star and computing the recovery fraction of injected small planets with
$P \leq \textbf{100}$ days. The average \ktwo{} detection sensitivity curve, along with the
$16^{\text{th}}$ and $84^{\text{th}}$ percentiles, are also included in Fig.~\ref{fig:senscurves}. The quality
of the pointing corrections within the \texttt{EVEREST} light curves can vary widely within our sample such
that there is considerably more variance in the \ktwo{} detection sensitivity relative to \kepler{.} Furthermore,
the average detection sensitivity is significantly reduced compared to \kepler{.} %The effect of the reduced
%sensitivity on the derived occurrence rates is discussed in Sect.~\ref{sect:occurrence}.
The reduced sensitivity is due in-part to the imperfect corrections of the reduced pointing accuracy and
to the limited time baseline of $\sim 80$ days in a typical \ktwo{} light curve compared to \kepler{.} 
Furthermore, we have not attempted to optimize the performance of \texttt{ORION} on \ktwo{} light curves 
beyond slight modifications to the algorithm's performance hyperparameters that were made to ensure the
detection of 52/53? confirmed \ktwo{} planets. The planet K2-21c (EPIC 206011691.02, $P=15.5$ days)
remains undetected by \texttt{ORION} because of the algorithm's requirement to discard putative signals
that are commensurate with other high S/N signals in the light curve. The presence of K2-21b at $P=9.32$ days
is within 1\% of a 5:3 period ratio with K2-21c and thus inhibits the identification of the 15.5 day signal
as being independent and planetary.


\subsection{Two-dimensional sensitivity maps}
The sensitivity curves depicted in Fig.~\ref{fig:senscurves} enable us to extend the visualization of the
detection sensitivity to two dimensions. Explicitly, we consider the detection sensitivity
$s_{nij}$ for each star (indexed by $n$) and as a function of $P$ and $r_p$ which are indexed by $i$ and
$j$ respectively. Consideration of the sensitivity in the $P-r_p$ plane will be needed to evaluate the
occurrence rates in that parameter space which is important for understanding the structure of
the radius valley around low mass stars due to the dependence of the efficiency of atmospheric loss on
both planet size and separation, regardless of the physical mechanism involved. 

We consider orbital periods $P \in [0.5,100]?$ days and planet radii $r_p \in [0.5,4]$ R$_{\oplus}$. At
each grid cell $nij$ we compute the average S/N within the cell and map that value to the detection sensitivity
using the data in Fig.~\ref{fig:senscurves}. The detection sensitivity maps for \kepler{} and \ktwo{,}
averaged over the index $n$, are shown in Fig.~\ref{fig:sensmap}.

 

\begin{figure*}
  \centering
  %\includegraphics[width=0.98\hsize]{figures/sensmap.png}
  \caption{Average detection sensitivity versus orbital period and planetary radius.
    The detection sensitivity maps averaged over \kepler{} stars (\emph{left panel)} and over \ktwo{} stars
    (\emph{right panel}) from our sample of low mass dwarf stars.} 
  \label{fig:sensmap}
\end{figure*}


\subsection{Survey Completeness}
Only transiting planets are detectable in transit surveys. To correct for the non-detection of otherwise
detectable but non-transiting planets we compute
the geometric transit probability for each star $n$ and at each grid cell $ij$ in the $P-r_p$ space to be

\begin{equation}
  p_{t,nij} = \frac{R_{s,n} + r_{p,j}}{a_{ni}}. \label{eq:ptransit}
\end{equation}

\noindent Note that we are only interested in the relative planet occurrence rate and therefore do not consider
constant scalar modifications to $p_{t,nij}$ from effects such as grazing transits or non-zero eccentricities 
\citep{barnes07b}.

The product of each star's detection sensitivity with its geometric transit probability yields completeness
maps as a function of $P$ and $r_p$. The average completeness maps for our \kepler{} and \ktwo{} stars are
shown in Fig.~\ref{fig:compmap}.

\begin{figure*}
  \centering
  %\includegraphics[width=0.98\hsize]{figures/compmap.png}
  \caption{Average completeness versus orbital period and planetary radius.
    Maps of the product of the detection sensitivity and geometric transit probability averaged over \kepler{} stars
    (\emph{left panel)} and over \ktwo{} stars (\emph{right panel}) from our sample of low mass dwarf stars.} 
  \label{fig:compmap}
\end{figure*}


\section{The Occurrence Rate of Small Close-in Planets around Low Mass Dwarf Stars} \label{sect:occurrence}
\subsection{Occurrence rates versus orbital period and planet radius}
The detection and validation of planets from the \kepler{} and \ktwo{} missions enable to measurement of the
occurrence rate of planets given the completeness corrections derived in Sect.~\ref{sect:completeness}.
For the index $i$ representing a planet's orbital period and $j$ representing the planetary
radius, the probability of detecting an integer number of planets within that grid cell ($k_{ij}$) around
$N_s$ stars is given by the binomial likelihood function

\begin{equation}
  \mathcal{L}_{nij}(k_{ij}|N_s,P_{nij}) = \binom{N_s}{k_{ij}} \prod_{n=1}^{N_s} P_{nij}^{k_{ij}} (1-P_{nij})^{N_s-k_{ij}}
  \label{eq:lnL}
\end{equation}

\noindent where

\begin{equation}
  P_{nij} = s_{nij} \cdot p_{t,nij} \cdot f_{ij},
  \label{eq:prob}
\end{equation}

\noindent is the probability of detecting a planet in the $ij$ grid cell around the $n^{\text{th}}$ star.
This quantity is dependent on the detection sensitivity $s_{nij}$, the transit probability $p_{t,nij}$, and the 
intrinsic occurrence rate of planets in the grid cell $ij$: $f_{ij}$.
The number of planet detections $k_{ij}$ is depicted in Fig.~\ref{fig:Ndet} for $i$ representing the
orbital period and insolation. Together with our calculations of $s_{nij}$ and $p_{t,nij}$ for the $N_s=17393$
low mass stars in our sample (Figs.~\ref{fig:sensmap}~and~\ref{fig:compmap}), and noting from Bayes theorem
that the posterior probability of $f_{ij}$ is 

\begin{equation}
  p(f_{ij}|N_s,s_{nij},p_{t,nij},k_{ij}) \propto \mathcal{L}_{nij}(k_{ij}|N_s,s_{nij},p_{t,nij},f_{ij}),
\end{equation}
  
\noindent modulo the coefficient of proportionality which we set to unity.

Firstly, recall that there are $\sim 5$ times more confirmed planets from \kepler{} than from \ktwo{}
in our sample
(see Fig.~\ref{fig:Ndet}) despite our sample having $\sim 3.5$ times more \ktwo{} stars 
(see Fig.~\ref{fig:stars}). These factors compound to produce a lower planet occurrence rate measured
with \ktwo{} than with \kepler{} as the reduced \ktwo{} detection completeness %relative to \kepler{}
(see Fig.~\ref{fig:compmap}) is insufficient to account for the lower measured
planet occurrence rates. The discrepancy instead arises from the disparate resources that have
been dedicated to the confirmation of planets from \kepler{} and \ktwo{.} The result being that
the number of confirmed planets existing within the set of \ktwo{} PCs is underestimated
by the number of PCs that have been reported as validated so far. We address this discrepancy
by scaling the cumulative occurrence rate measured by \ktwo{} in our sample to that of \kepler{.}
In this way, we are assuming that the planet populations studied by each mission are inherently
identical despite existing within distinct stellar populations within the galaxy.

The maximum a-posteriori (MAP) map of $f_{ij}$ is depicted in Fig.~\ref{fig:fmap}.
The persistence of the radius valley from Sun-like to the low mass stars in our sample is clearly
evident. Distinct peaks are separated along the planetary radius axis and span
$\sim 0.9-1.4$ R$_{\oplus}$ and $\sim 1.9-2.3$ R$_{\oplus}$ respectively.
Note however that the lower limit on the former peak nearly
approaches the region in which the \kepler{} sensitivity falls below 10\% and the $f$ values become
unreliable. %The local peaks between $\sim 2.0-2.4$ R$_{\oplus}$ and separated in period space
%at $\sim 10$ and $\sim 50$ days may not be real and is instead an artifact of the binning depicted in
%Fig.~\ref{fig:fmap}.
The occurrence rates also highlight the relative dearth of planets larger than $\sim 3$ R$_{\oplus}$
including the Neptunian desert at short orbital periods \citep{lundkvist16,mazeh16}. The large scale
structure of the measured occurrence rates are also broadly consistent with previous investigations
of the planet population around low mass \kepler{} stars \citep{dressing13,dressing15a,gaidos16} such
as the prominence of planets $\lesssim 2$ R$_{\oplus}$ with $P \sim 10-60$ days and a cumulative
occurrence rate of 2.87? planets per star.

\begin{figure*}
  \centering
  %\includegraphics[width=.7\hsize]{figures/fmap_Msgt0d0_xbin39_ybin26.png}
  \caption{Planet occurrence rate versus orbital period and planetary radius. The maximum a-posteriori occurrence
    rate map calculated from the population of confirmed planets from \kepler{} (\emph{circles}) and
    \ktwo{} (\emph{diamonds}) around low mass dwarf
    stars. Overplotted are the empirical locations of the radius valley around FGK stars 
    characterized via asteroseismology \citep[\emph{dashed line},][]{vaneylen18} and the approximate radius
    valley around early-M to mid-K dwarfs \citep[\emph{dotted line},][]{wu19}.}
  \label{fig:fmap}
\end{figure*}

The location and slope of the radius valley (i.e. $\text{d}\log{r_p}/\text{d}\log{P})$ are broadly consistent
with the valley structure measured from the empirical planet population of FGK stars characterized via
asteroseismology \citep{vaneylen18}. \cite{wu19} also provided a visual approximation to the location of the
radius valley around stars with $M_s \in [0.5,0.76]$ M$_{\odot}$ in their \gaia{-}\kepler{} sample. However
we find the location of the terrestrial-sized planet peak to exist at longer $P \sim 30$ days compared to
$\sim 5$ days \citep[c.f. Fig. 2,][]{wu19}. The discrepancy likely stems from differences in the methods of
correcting for survey incompleteness. Recall that in this study the detection sensitivity is computed for each
star using it's unique completeness products from the \kepler{} pipeline whereas \cite{wu19} adopt piecewise
completeness levels of 10, 50, or 90\% complete as a function of $P$ and $r_p$ from \cite{zhu18b}.  

Lastly, we note that the radius valley as a function of $P$ does not appear to be completely void of planets.
This may present evidence of the efficiency of the gap clearing mechanism around low mass stars and is
discussed further in Sect.~\ref{sect:void}.

\subsection{Occurrence rates versus planet radius}
Next we marginalize over $P$ and compute the one-dimensional occurrence rate of small, close-in planets
as a function of $r_p$. The resulting occurrence rates are shown in Fig.~\ref{fig:rphist} in which the bimodal
distribution of planet sizes is again clearly discernible in the MAP occurrence rates.
The uncertainties on each $f_j$ bin are computed from the $16^{\text{th}}$ and $84^{\text{th}}$ percentiles
of the $f_j$ posterior. We ignore the occurrence rates in bins with $r_p\lesssim 1$ R$_{\oplus}$ where our
detection sensitivity is poor.

  
\begin{figure*}
  \centering
  %\includegraphics[scale=.8]{figures/rphist_double.png}
  \caption{Occurrence rate of planets as a function of size. \emph{Upper panel}:
    histogram depicting the relative occurrence
    rate of close-in planets with orbital periods $<100$ days derived from the population of confirmed
    \kepler{} and \ktwo{} planets around low mass stars. The bimodal distribution of planet radii peaking
    at 1.12 and 2.07 R$_{\oplus}$ highlights the presence of the radius valley around low mass stars centered at
    1.54 R$_{\oplus}$. Uncertainties in the planet occurrences follow from binomial statistics and are limited by
    the relatively small number of confirmed planets around low mass stars from \kepler{} and \ktwo{.} 
    The measured occurrence rates below $\sim 1$ R$_{\oplus}$ (shown in \emph{grey}) should be ignored due to
    the poor detection sensitivity. \emph{Lower panel}: identical occurrence rates as in the upper panel 
    accompanied by the same occurrence rates with finer radius bins.
    The corresponding occurrence rate uncertainties
    are too large to robustly infer the presence of features but the bimodal structure continues
    to be exhibited in the maximum likelihood occurrence rates. The shaded regions highlight our approximate
    definitions of terrestrial planets ($r_p \in [0.97,1.57]$ R$_{\oplus}$), down to reasonable sensitivity limits,
    and gaseous planets ($r_p > 1.57$ R$_{\oplus}$) around low mass stars. Note the outer limit of 2.5 R$_{\oplus}$
    is arbitrary.}
  \label{fig:rphist}
\end{figure*}

From the bimodal distribution we highlight the approximate radii likely corresponding to planets with terrestrial
bulk compositions ($r_p\lesssim 1.55$ R$_{\oplus}$) versus planets with significant size fractions in gaseous
envelopes ($r_p\gtrsim 1.55$ R$_{\oplus}$) around
low mass stars. Also depicted in Fig.~\ref{fig:rphist} is $f_j$ with a binning scale twice that of the primary
$f_j$ depiction (i.e. 0.06 R$_{\oplus}$ compared to 0.13 R$_{\oplus}$).
With this adopted fine binning the fractional uncertainties on $f_j$ in each bin are sufficiently
large to eliminate the significance of the distinct bimodal peaks. Despite this, the bimodality in the MAP
occurrence rates continues to persist with the location of the valley features only being marginally affected.
We interpret this as further evidence for the existence of the radius valley in the close-in planet population
around low mass stars. 


\subsection{Inclusion of supplemental K2 planet candidates}
In an attempt to improve the counting statistics in the occurrence rate calculations,
we consider an enlarged planet population. This population is the union of our confirmed planet sample with
a set of additional planet candidates (PCs) from the \ktwo{}
mission. Specifically, we consider the set of PCs reported by \cite{kruse19}
%% could add more studies here
from \ktwo{} campaigns 0-8 that includes 126 PCs not already included in our sample of confirmed planets
and orbiting stars contained within our stellar sample.

By definition, we cannot identify which PCs are true planets of interest for this study and
which PCs are instead produced by an astrophysical false positive. 
The inclusion of \ktwo{} PCs therefore requires that we account for sample contamination by false positives
probabilistically. We do so by considering
a number of studies from the literature that perform a transiting planet search in \ktwo{,} from any subset of its campaigns,
and present attempts at validating their uncovered PCs statistically based on follow-up observations
\citep{montet15,crossfield16b,dressing17,hirano18,livingston18a,mayo18}. Each of these studies utilized some combination of
ground-based photometry to validate planet ephemerides, reconnaissance
spectroscopy to identify spectroscopic binaries, and speckle or AO-assisted imaging to search for nearby stellar companions to
their PC host stars. Each of the aforementioned studies used their respective set of follow-up observations together with the
statistical validation tool \texttt{vespa} \citep{morton12,morton15} to statistically classify their PCs as either a validated
planet (VP)\footnote{We treat validated and confirmed planets as equivalent dispositions.},
a false positive (FP), or some other inconclusive disposition (e.g. remains a PC). The FP rate around cool
stars (\teff{} $< 4700$ K) from each study is estimated by calculating the ratio of the number of reported FPs to
the total number of FPs and VPs. Notably, \cite{crossfield16b} showed that the FP rate in their \ktwo{} sample is dependent
on the measured planet radius as giant PCs have a larger likelihood of
being a FP. Hence, we focus on PCs with $r_p<4$ R$_{\oplus}$ when deriving FP rates.

The resulting FP rates are reported in Table~\ref{tab:FP}. Half of the studies do not find any probable
FP signals among the small PCs orbiting cool stars in their samples. In such cases only upper limits on the FP rate
can be derived which all agree that the FP rate is $\lesssim 20$\% at 95\%. The remaining studies
each detect at least one FP such that a non-zero maximum likelihood FP rate is measured. Their average FP rate is 5.7\%
which is also in agreement with the derived upper limits from the aforementioned studies.
We proceed by constructing $10^3$ realizations of the planet population
that includes confirmed planets from both \kepler{} and \ktwo{} plus a subset of the 126 \ktwo{} PCs from
\cite{kruse19}. The subset of included PCs are randomly sampled from the full PC set
according to the adopted FP rate such that each realization contains $0.943\cdot 126 \approx 119$ PCs.

\begin{deluxetable}{lccc}
\tabletypesize{\small}
\tablecaption{\ktwo{} false positive rates for small planets around cool stars\label{tab:FP}}
\tablehead{Reference & $N_{\text{FP}}$ & $N_{\text{VP}}$ & FP rate [\%]}
\startdata
\cite{montet15}\tablenotemark{a} & 0 & 8 & $<30.7$ \\
\cite{crossfield16b} & 2 & 39 & $4.9^{+6.0}_{-1.4}$ \\
\cite{dressing17} & 2 & 34 & $5.6^{+6.4}_{-2.0}$ \\
\cite{hirano18}\tablenotemark{a} & 0 & 16 & $<19.5$ \\
\cite{livingston18a}\tablenotemark{a}  & 0 & 14 & $<21.0$ \\
\cite{mayo18}\tablenotemark{b} & 1 & 14 & $6.7^{+12.4}_{-2.0}$
\enddata
\tablecomments{Within each study we only consider PCs with $r_p <4$ R$_{\oplus}$ and orbiting cool stars with \teff{} $<4700$ K. FP: false positive. VP: validated planet.}
\tablenotetext{a}{These studies do not detect any FPs such that the reported FP rate upper limit is represented by its 95\% confidence interval.}
\tablenotetext{b}{\cite{mayo18} did not explicitly classify their non-validated planets as FPs so we define FPs within their sample as any PC whose false positive probability exceeds 10\%.}
\end{deluxetable}



The effect of including PCs in the planet sample on the derived occurrence rates are compared to the results
from only considering confirmed planets in Fig.~\ref{fig:rphistPCs}. Again we scale the \ktwo{} occurrence rates
to those from \kepler{} such that the cumulative occurrence of close-in planets with $r_p \leq 4$ R$_{\oplus}$
is identical between the two planet populations. The radius valley continues to be resolved in
the MAP occurrence rates. Furthermore, the addition of PCs reduces the median $f_j$ uncertainty among planets
with $r_p>1$ R$_{\oplus}$ from
0.0216 to 0.0186 planets per star (i.e. $\sim 15$\% improvement). However, the partial filling of the gap is
increased as the contrast between the maximum $f_j$ of the terrestrial planet peak ($r_p \sim 1.3$ R$_{\oplus}$)
and the minimum $f_j$ of the valley ($r_p \sim 1.6$ R$_{\oplus}$) decreases from 0.070 R$_{\oplus}$ ($3.2\sigma$)
to 0.054 ($2.9\sigma$).

\begin{figure*}
  \centering
  %\includegraphics[scale=.8]{figures/rphist_kruse.png}
  \caption{Comparison of occurrence rates with and without planet candidates included. \emph{Left panel}:
    same as Fig.~\ref{fig:rphist}. \emph{Right panel}: histogram depicting the relative occurrence
    rate of close-in planets with orbital periods $<100$ days derived from the population of confirmed
    planets from \kepler{} and \ktwo{} and supplemented by 119 PCs
    around low mass stars from \cite{kruse19}. The radius valley continues to be resolved with
    the inclusion of PCs which improve the median uncertainty on the occurrence rate bins although the
    gap becomes less prominent with numerous PCs partially filling the valley.}
  \label{fig:rphistPCs}
\end{figure*}


\section{Evolution of the radius valley around low mass stars} \label{sect:models}
\subsection{Slope of the radius valley}
%Similarly to Fig.~\ref{fig:fmap},
Fig.~\ref{fig:fmapF} shows the two-dimensional occurrence rate of planets in the $F-r_p$ space for
our planet sample as well as for the close-in \kepler{} planets from \cite{martinez19}.
In this space we calculate the slope of the radius valley with $F$ and compare the measured value 
to model predictions of the planet size marking the transition from terrestrial to gaseous planets
versus insolation. We measure the slope by resampling $10^3$ planet populations from the inverse
occurrence rates (and their fractional uncertainties at each point $ij$) over the domains
$F\in [1,30]$ F$_{\oplus}$ and $r_p\in [1,2.5]$ R$_{\oplus}$.
The domains are chosen to neglect regions far from the radius valley which would otherwise dominate the
inverse occurrence rates. The number of sampled planets in the $\log{F}-r_p$ space is then fit with a
linear function as depicted in Fig.~\ref{fig:fmapF}. Over the $10^3$ realizations of the resampled planet
population, we measure an average slope and standard deviation to be
$\text{d}r_p / \text{d}\log{F} = -0.123\pm 0.054$. Similarly, repeating this exercise in the $P-r_p$
space yields $\text{d}r_p / \text{d}\log{P} = 0.085\pm 0.074$.


\begin{figure*}
  \centering
  \includegraphics[width=.48\hsize]{figures_tmp/fmapF_Msgt0d0_xbin39_ybin26_insolation.png}
  \includegraphics[width=.48\hsize]{figures_tmp/fmapF_martinez/fmapF_martinez001.png}
  \caption{Planet occurrence rates versus insolation and planetary radius around low mass and Sun-like stars.
    \emph{Left panel}: the maximum a-posteriori occurrence
    rate map calculated from the population of confirmed planets from \kepler{} (\emph{circles}) and
    \ktwo{} (\emph{diamonds}) around low mass dwarf
    stars. Overplotted in black are model predictions of the transition from terrestrial to gaseous
    planets in the following scenarios: core-powered mass loss \citep{gupta19b}, photoevaporation
    \citep{lopez18}, and gas-poor formation \cite{lopez18}. We measure the slope of the radius valley
    to be $\text{d}r_p / \text{d}\log{F} = -0.123\pm 0.054$ which is broadly consistent with predictions
    from gas-poor formation of terrestrial planets. \emph{Right panel}: the occurrence rate map of close-in
    \kepler{} planets around Sun-like stars \citep{martinez19}. Note the unique $F$ and $r_p$ scales compared
    to the left panel. The measured slope of } 
  \label{fig:fmapF}
\end{figure*}

The negative slope $\text{d}r_p / \text{d}\log{F} = -0.123$ indicates that the location of the radius valley
drops to smaller planet radii with increasing insolation (i.e. with decreasing orbital separation). This behavior
is broadly consistent with models of the formation small terrestrial planets in a gas-poor environment
\citep{lee14,lee16,lopez18} leading to the transition from rocky to gaseous planets occurring at larger planet
radii at larger orbital separations. The theoretical scaling of the transition
radius with insolation in the gas-poor formation scenario is $r_{p,valley} \propto F^{-0.08}$
\citep{lopez18} which is
consistent with our measured scaling of $-0.123\pm 0.054$. This result is inconsistent with the
models of photoevaporation and core-powered mass loss which each predict an increasing central valley
radius with increasing insolation ($r_{p,valley} \propto F^{0.11}$; \citealt{lopez18},  
$r_{p,valley} \propto F^{0.10}$; \citealt{gupta19b}).

The negative slope in the radius valley around low mass stars differs from the trend seen around
Sun-like stars \citep{fulton17}. Fig.~\ref{fig:fmapF} compares the occurrence rates of small close-in
planets around each stellar population. Multiple studies have concluded that the location of the
radius valley around Sun-like stars drops to smaller planet sizes with decreasing insolation
\citep[e.g.][]{vaneylen18,martinez19}. Conversely, Fig.~\ref{fig:fmapF} indicates that the opposite
may be true around low mass stars. This observational evidence may elucidate changes in the dominant
physical mechanisms that sculpt the valley around different host spectral types.


\subsection{Planet populations versus stellar mass} \label{sect:Msbin}
In addition to calculating the occurrence rates $f_{ij}$ among our full stellar sample, we also consider
the evolution of the planet population in unique host stellar mass bins. Fig.~\ref{fig:rphistcomp} shows
the MAP $f_{ij}$ maps in $P-r_p$ space and the marginalized $f_j$ distributions in four stellar mass bins:
our full stellar sample
($M_s \in [0.08,0.93]$ M$_{\odot}$), the massive half of the sample ($M_s>0.65$ M$_{\odot}$),  
the low mass half of the sample ($M_s<0.65$ M$_{\odot}$), and a subset of the latter focusing on
increasingly lower mass stars ($M_s<0.42$ M$_{\odot}$). The statistically significant resolution of the
radius valley in the $f_j$ occurrence rates is only accomplished with the full stellar sample. The reduction
of the sample size in the three remaining mass bins inflates the $f_j$ uncertainties such that the valley
is seen at $<1\sigma$ and hence not detected. However, the characteristic bimodality is exhibited in the
MAP $f_{ij}$ of the full and massive half stellar samples. Furthermore, their $f_{ij}$ structures are similar
as the majority of our full planet sample orbit stars more massive than the median stellar mass of 0.65
M$_{\odot}$ ($\sim 62$\%).

\begin{figure*}
  \centering
  %\includegraphics[width=.98\hsize]{figures/rphist_compMs.png}
  \caption{2D and 1D planet occurrence rates in various stellar mass bins. \emph{Top panels}: planet occurrence
    rate maps as a function of orbital period and planet radius. \emph{Bottom panels}: distributions of the relative
    planet occurrence rate as a function of planet size. Note the differing occurrence rate scales.
    Each column corresponds to a unique cut in stellar masses
    which represent the full stellar sample ($M_s \in [0.08,0.93]$ M$_{\odot}$), the early half of the stellar sample
    ($M_s \in [0.65,0.93]$ M$_{\odot}$), the late half of the stellar sample ($M_s \in [0.08,0.65]$ M$_{\odot}$),
    and the low mass bin ($M_s \in [0.08,0.42]$ M$_{\odot}$) depicting a subset of the late half of the stellar sample.
    %As the stellar mass range is decreased the statistics become poorer which muddles the detection of the radius
    %valley in any subset other than the full sample.}
    The relative occurrence of terrestrial to gaseous planets appears to increase around lower mass stars.}
  \label{fig:rphistcomp}
\end{figure*}

In considering stars less massive than 0.65 M$_{\odot}$, the gaseous planet peak begins to diminish relative
to the terrestrial-sized planets. As evidenced in the MAP $f_j$ distribution around stars with
$M_s \in [0.08,0.65]$ M$_{\odot}$, the radius valley might persist around 1.6 R$_{\oplus}$ but the gaseous planet
peak does not appear distinct from the terrestrial planet peak in the MAP $f_{ij}$ map. That is that the relative
frequency of terrestrial to gaseous planets appears to increase significantly around M dwarfs compared to the more
massive K dwarfs. This feature is further accentuated around the lowest mass stars in our sample ($<0.42$
M$_{\odot}$) for which terrestrial-sized planets clearly dominate the distribution of close-in planets. The relative
frequency of terrestrial to gaseous planets in each stellar bin are reported in Table~\ref{tab:rel}
for fixed definitions of $r_p \in [1,1.6]$ R$_{\oplus}$ and $r_p \in [1.6,2.5]$ R$_{\oplus}$ respectively. The inner
limit of 1 R$_{\oplus}$ restricts this analysis to where the detection sensitivity is still informative. The outer
limit of 2.5 R$_{\oplus}$ is chosen such that the full width at half maximum of the gaseous planet peak in the
$f_j$ distribution from the full stellar sample (Fig.~\ref{fig:rphist}) is approximately identical for each peak.

\capstartfalse
\begin{deluxetable}{cccc}
\tabletypesize{\small}
\tablecaption{Relative occurrence rates of close-in rocky and non-rocky planets around low mass stars\label{tab:rel}}
\tablehead{Stellar mass & $f_{\mathrm{rocky}}$ & $f_{\mathrm{non-rocky}}$ & $f_{\mathrm{rocky}}/f_{\mathrm{non-rocky}}$ \\
range $[\mathrm{M}_{\odot}]$ & $r_p \in [1,1.6]$ & $r_p \in [1.6,2.5]$ &}
\startdata
$[0.08,0.90]$ & $0.68\pm 0.07$ & $1.02\pm 0.08$ & $0.66\pm 0.09$ \\
$[0.63,0.90]$ & $0.69\pm 0.11$ & $1.28\pm 0.16$ & $0.54\pm 0.11$ \\
$[0.08,0.63]$ & $1.10\pm 0.16$ & $1.02\pm 0.16$ & $1.08\pm 0.23$ \\
$[0.08,0.42]$ & $1.64\pm 0.43$ & $0.19\pm 0.09$ & $8.46\pm 4.62$
\enddata
\end{deluxetable}
\capstarttrue


The values in Table~\ref{tab:rel} indicate the significant increase in the relative occurrence of terrestrial
planets with decreasing stellar mass that is also illustrated in Fig.~\ref{fig:rphistcomp}. This is evidenced by
our measurements showing that gaseous planets are nearly twice as common as terrestrial planets
around mid to late K dwarfs ($M_s \in [0.65,0.93]$ M$_{\odot}$) while the relative frequency approaches unity
around M dwarfs ($M_s \in [0.08,0.65]$ M$_{\odot}$). Focusing on the lowest stellar mass bin considered,
terrestrial planets become much more prominent as they outnumber gaseous planets by a factor of
$8.46\pm 4.62$. This result is broadly
consistent the calculations of \cite{hardegree19} who find that terrestrial-sized 
planets with $r_p \in[0.5,1.5]$ R$_{\oplus}$ are about $4-5$ times as common as gaseous planets ()
around M3-5.5 dwarfs ($M_s \in [0.12,0.38]$ M$_{\odot}$). Thus we have supporting evidence 

\subsection{Dependence of radius valley features on stellar mass}
Here we measure the locations and uncertainties of features in the radius valley in each of the stellar
mass bins considered in Sect.~\ref{sect:Msbin}. In each of the four $f_{ij}$ occurrence rate maps shown
in Fig.~\ref{fig:rphistcomp}, we measure the frequency-weighted central radius of the terrestrial planet peak,
the gaseous planet peak, and the radius valley. We quantify the uncertainties in the feature locations by
marginalizing over the following hyperparameters that can directly affect the inferred radius of each feature:
the $f_{ij}$ smoothing parameter, the minimum detection sensitivity,
the $P$ bin width, the $r_p$ bin width, along with the imposed upper and
lower $P$ and $r_p$ limits on each peak. These upper and lower $r_p$ limits are defined based on the visual
inspection of the $f_{ij}$ maps in Fig.~\ref{fig:rphistcomp} and are used to demarcate the assumed
boundaries of each peak, and by extension the valley separating the peaks.
For example, if the boundaries on the terrestrial peak are set to 1-50 days and 0.8-1.4 R$_{\oplus}$ then only
the planet occurrence rates over that subset of the $P-r_p$ parameter space are considered when calculated the
$f_j$-weighted peak radius. The boundaries are listed in Table~\ref{tab:bounds}.
In practice we derive $10^3$ realizations of each $f_{ij}$ map with each realization having a unique set of
hyperparameters. The resulting $f_{ij}$ maps are marginalized over $P$ and the $f_j$-weighted radius of
each peak is computed over the domain bounded by the relevant hyperparameters (see Table~\ref{tab:bounds}).
The same is done for the radius valley using the inverse occurrence rates.

\capstartfalse
\begin{deluxetable*}{ccccccc}
\tabletypesize{\small}
\tablecaption{Assumed boundary ranges on the locations of radius valley features\label{tab:bounds}}
\tablehead{Stellar mass & $\log{P}$ lower & $\log{P}$ upper & Terrestrial & Terrestrial  & Gaseous  & Gaseous \\
  range & boundary & boundary & peak lower $r_p$ & peak upper $r_p$ & peak lower $r_p$  & peak upper $r_p$  \\
  $[\mathrm{M}_{\odot}]$ & $[\text{days}]$ & $[\text{days}]$ & boundary $[\text{R}_{\oplus}]$ & boundary $[\text{R}_{\oplus}]$ & boundary $[\text{R}_{\oplus}]$ & boundary $[\text{R}_{\oplus}]$}
\startdata
$[0.08,0.90]$ & $\mathcal{U}(\log{0.5},\log{2})$ & $\mathcal{U}(\log{50},\log{100})$ & $\mathcal{U}(0.8,1)$ & $\mathcal{U}(1.2,1.5)$ & $\mathcal{U}(1.6,1.9)$ & $\mathcal{U}(2.3,2.5)$ \\
$[0.63,0.90]$ & $\mathcal{U}(\log{0.5},\log{2})$ & $\mathcal{U}(\log{50},\log{100})$ & $\mathcal{U}(0.8,1)$ & $\mathcal{U}(1.3,1.5)$ & $\mathcal{U}(1.8,2)$ & $\mathcal{U}(2.4,2.7)$ \\
$[0.08,0.63]$ & $\mathcal{U}(\log{0.5},\log{2})$ & $\mathcal{U}(\log{50},\log{100})$ & $\mathcal{U}(0.6,0.9)$ & $\mathcal{U}(1.2,1.4)$ & $\mathcal{U}(1.8,2)$ & $\mathcal{U}(2.1,2.3)$ \\
$[0.08,0.42]$ & $\mathcal{U}(\log{0.5},\log{2})$ & $\mathcal{U}(\log{50},\log{100})$ & $\mathcal{U}(0.5,0.7)$ & $\mathcal{U}(1.3,1.4)$ & $\mathcal{U}(1.7,1.8)$ & $\mathcal{U}(1.8,2)$
\enddata
\tablecomments{The $r_p$ boundaries on the radius valley are given implicitly by the upper $r_p$ limit on the terrestrial peak and the lower $r_p$ limit on the gaseous peak.}
\end{deluxetable*}
\capstarttrue


The resulting locations of each radius peak and the valley are depicted in Fig.~\ref{fig:rpvMs} as a function of
stellar mass and given explicitly in Table~\ref{tab:rpvMs}. The depicted $M_s$ values are given by the median and 
with uncertainties given by the $16^{\text{th}}$ and  $84^{\text{th}}$ percentiles. In computing the
feature locations we assume that the radius valley is present in all stellar mass bins despite the waning
evidence for its existence around stars with $M_s \lesssim 0.4$ M$_{\odot}$ (see Fig.~\ref{fig:rphistcomp}).
Our measured feature radii are compared to those measured in \cite{fulton18} around Sun-like stars with
$M_s <0.97$, $\in [0.97,1.11]$, and $>1.11$ M$_{\odot}$.
%A number of features are exhibited in Fig.~\ref{fig:rpvMs}.
Most notably, the feature locations obtained from our full stellar sample continue to the trend of monotonically
decreasing  to smaller $r_p$ with decreasing $M_s$. The slopes of this decrease for the terrestrial and gaseous
planet peaks are $\text{d}r_{p,\text{terr}} / \text{d}M_s = 0.40$ and $\text{d}r_{p,\text{gas}} / \text{d}M_s = 0.97$
respectively indicating that the most common size of
gaseous planet decreases more steeply with $M_s$ than the typical size of terrestrial planets. We interpret this
as being indicative of the effective disappearance of gaseous planets around increasingly lower mass stars
(see Table~\ref{tab:rel}). Furthermore, the reduced slope of the terrestrial peak may be indicative of
a characteristic planetary core size of $\approx 1$ R$_{\oplus}$. 
%apparent convergence of the 

\begin{deluxetable}{cccc}
\tabletypesize{\small}
\tablecaption{Radius valley features versus stellar mass\label{tab:rpvMs}}
\tablehead{Stellar mass & Terrestrial peak & Radius valley & Gaseous peak \\
$[\mathrm{M}_{\odot}]$ & $[\text{R}_{\oplus}]$ & $[\text{R}_{\oplus}]$ & $[\text{R}_{\oplus}]$}
\startdata
$0.651^{+0.058}_{-0.096}$ & $1.118^{+0.151}_{-0.148}$ & $1.543^{+0.160}_{-0.160}$ & $2.068^{+0.211}_{-0.205}$ \\
$0.684^{+0.040}_{-0.035}$ & $1.154^{+0.205}_{-0.239}$ & $1.647^{+0.207}_{-0.215}$ & $2.197^{+0.301}_{-0.256}$ \\
$0.500^{+0.097}_{-0.146}$ & $1.036^{+0.297}_{-0.308}$ & $1.599^{+0.340}_{-0.352}$ & $2.048^{+0.191}_{-0.199}$ \\
#$0.343^{+0.057}_{-0.092}$ & $1.017^{+0.700}_{-0.807}$ & $1.548^{+0.515}_{-0.496}$ & $1.815^{+0.260}_{-0.192}$
$0.343^{+0.057}_{-0.092}$ & $1.017^{+0.700}_{-0.807}$ & $1.548^{+0.515}_{-0.496}$ & -
\enddata
\tablecomments{As depicted in Fig.~\ref{fig:rpvMs}.}
\end{deluxetable}


\begin{figure*}
  \centering
  %\includegraphics[width=0.85\hsize]{figures/rpvMsFULL_KepK2.png}
  \caption{Evolution of the radius valley features with stellar mass. \emph{Solid markers}:
    the occurrence rate-weighted locations of the terrestrial planet peak (\emph{red markers}), the
    radius valley (\emph{black markers}) and the gaseous planet peak (\emph{blue markers})
    as a function of host stellar mass. Measurements around Sun-like stars with $M_s>0.8$ M$_{\odot}$ are
    retrieved from \cite{fulton18} (\emph{open circles}). Feature radii from our full sample with
    $M_s = 0.651^{+0.058}_{-0.096}$ M$_{\odot}$ are depicted as \emph{filled circles}. \emph{Filled squares}
    depict the feature radii from partitioning our sample into three $M_s$ bins: $M_s \in [0.65,0.93]$ M$_{\odot}$,
    $M_s \in [0.08,0.65]$ M$_{\odot}$, and $M_s \in [0.08,0.42]$ M$_{\odot}$. Uncertainties on the
    peak and valley locations are derived by sampling the measured occurrence rates and their uncertainties along
    with samples of the hyperparameters controlling map smoothing, minimum detection sensitivity, planet parameter
    binning, and the assumed feature ranges in $P$ and $r_p$. The green
    curves represent theoretical predictions for the evolution of the radius valley with stellar mass based on
    physical models gas-poor terrestrial planet formation (\emph{dotted}; \citealt{lopez18}),
    core-powered atmospheric mass loss
    with an empirical mass-luminosity relation (\emph{solid}; \citealt{gupta19b}), a constant mass-luminosity
    relation (\emph{dash-dotted}; \citealt{gupta19b}), and photoevaporation (\emph{dashed}; \citealt{wu19}).
    The models only predict scaling relations with $M_s$ and as such are anchored to the measured valley location
    at $M_s \sim \text{M}_{\odot}$.}
  \label{fig:rpvMs}
\end{figure*}

Models of the formation of the radius valley based upon photoevaporation \citep{wu19}, gas-poor formation \citep{lopez18},
and core-powered mass loss \citep{gupta19b} all make explicit predictions for the evolution of the radius valley location
with stellar mass. Predictions from the core-powered mass loss scenario are dependent on the stellar mass-luminosity
relation (MLR) $L_s \propto M_s^{\alpha}$. In Fig.~\ref{fig:rpvMs}, we consider cases with a constant MLR with $\alpha=5$
\citep{gupta19b} and with the empirically-derived piecewise MLR from \cite{eker18}. All models predict a decreasing
radius valley with decreasing stellar mass but differ in their slopes. 
At the median stellar mass of our full stellar sample ($0.65$ M$_{\odot}$), the measured location of the radius valley is
$1.54\pm 0.16$ R$_{\oplus}$. This value favors a steep $\text{d}r_p / \text{d}M_s$ slope although we are unable to distinguish
between competing physical models given the uncertainty in the location. Fortunately, the model predictions continue to diverge
with decreasing stellar mass. As such, measurements of the feature locations in decreasing $M_s$ bins can be used to
rule out the operation of certain physical mechanisms in the low stellar mass regime. Athough the trend of decreasing
median feature radii with decreasing stellar mass is upheld, the poor counting statistics in the reduced $M_s$ bins
prevent any signficant inference regarding the relative strength of the competing physical mechanisms.

\section{Discussion \& Conclusions} \label{sect:conclusion}
\subsection{Improving constraints on the occurrence rate of small close-in planets orbiting mid-M dwarfs}
The issue of having insufficient information to distinguish between photoevaporation, core-powered mass loss, and gas-poor
formation around low mass stars can be addressed with two steps. Firstly, by expanding the low mass stellar sample in transiting
searches for small close-in planets and secondly, by quantifying the detection sensitivity in those searches. 
NASA's Transiting Exoplanet Survey Satellite \citep[\tess{;}][]{ricker15} will provide hundreds of new transiting planet
discoveries in the vicinity of the radius valley \citep{barclay18}. \tess{} is particularly well-suited to the discovery of
close-in planets around low mass stars down to M5V ($M_s\sim 0.16$ M$_{\odot}$) due to its red bandpass (600-1000 nm)
and its high cadence (2 minute) observations of 200,000-400,000 stars over $\sim 94$\% of the sky following its recently
approved extended mission.

The \tess{} primary mission, lasting one year, has been ongoing since July 2018. 
Based on the mission's performance at the time of writing, we can calculate the number of stars required to be observed
by \tess{} to enable robust conclusions regarding the nature of the radius valley down to low mass stars. These calculations
proceed by noting that based on binomial statistics, the measurement uncertainty on the feature locations scales as
$\sqrt{N_sP(1-P)}$ where $N_s$ is the number of observed stars and $P$ is the probability of detecting a small close-in planet
given the detection sensitvity, its transit proability, and its inherent rate of occurrence (see Eq.~\ref{eq:prob}).
Through sectors 1-14, \tess{} has observed $N_{s,\text{TESS}} = 23051$ stars less massive than 0.4 M$_{\odot}$
from its Candidate Target List \citep[CTL;][]{stassun19} with 2 minute cadence.
Among these stars, the Science Processing Operations Center \citep[SPOC;][]{jenkins16,twicken18,li18} has reported
three objects of interest spanning the radius valley between $1.4-1.6$ R$_{\oplus}$. Assuming a 0\% false positive rate
among these planet candidates and the same MAP occurrence rate as measured with \kepler{}
($f_{\text{valley}}\approx 0.19$ planets per star), we find the probability of \tess{} to detect a transiting planet
spanning the radius valley around a star with $M_s<0.4$ M$_{\odot}$ to be 
$P_{\text{valley,TESS}}=1.30 \times 10^{-4}$. We can compare these to the \kepler{} values of $N_{s,\text{Kep}}=33$ and
$P_{\text{valley,Kep}}=8.56\times 10^{-3}$ to scale the uncertainty on $f_{\text{valley}}$, and hence on the valley radius, 
as an increasing number of low mass stars are observed with \tess{.}

The resulting improvement in the measurement precision of the radius valley is shown in Fig.~\ref{fig:improve}. ??


\begin{figure*}
  \centering
  %\includegraphics[width=0.98\hsize]{figures_tmp/rpuncertainty_TESS.png}
  \caption{Expected improvement in the measurement precision of the radius peak and valley locations with additional
    confirmed planets around stars with $M_s\sim 0.3$ M$_{\odot}$. Assuming a fixed detection probability (\textbf{dont do this}),
    the shaded
    regions depict the degree of improvement in the upper and lower limits on location of the super-Earth peak
    (\emph{blue}), the radius valley (\emph{green}), and the sub-Neptune peak (\emph{red}) as additional planets are
    detected by missions like TESS. For stars with $M_s\sim 0.3$ M$_{\odot}$, model predictions of the valley location
    from photoevaporation and core-powered mass loss differ by $\sim 0.4$ R$_{\oplus}$ (\emph{dashed horinzontal line}).
    Based on the performance of TESS to-date, the expected time to confirm such planets is parameterized as a linear
    function of time and is depicted on the secondary x-axis.}
  \label{fig:improve}
\end{figure*}

\subsection{Implications for RV planet searches around low mass stars}
Many existing and up-coming RV spectrographs
will be partially focused on characterizing the masses of planets spanning the radius valley in order to
improve our physical understanding of the nature of those planets. The subset of those spectrographs operating
in the near-IR, in particular, will focus heavily on M dwarf planetary systems
(e.g. CARMENES; \citealt{quirrenbach14}, HPF; \citealt{mahadevan12}, IRD; \citealt{kotani14},
NIRPS; \citealt{bouchy17}, SPIRou; \citealt{donati18}). In defining target samples
that are equally complete on either side of the radius valley, it is critically important that the transition
location between terrestrial and gaseous planets is known. In our full stellar sample which includes mid to
late K dwarfs, the measured radius valley location is $1.54\pm 0.16$ R$_{\oplus}$ although we remind the reader
that the exact value is dependent on the planet's separation (see Fig.~\ref{fig:fmapF}). A consistent value of
$1.55^{+0.52}_{-0.50}$ is also recovered, albeit with reduced signficance, around stars later than about M2.5V.
This value is slightly lower than the valley locations measured around Sun-like stars with $M_s \sim 1.2$
M$_{\odot}$ of $\sim 1.9$ R$_{\oplus}$ and  $M_s \sim 0.85$ M$_{\odot}$ of $\sim 1.7$ R$_{\oplus}$ \citep{fulton18}.

\subsection{Imperfect clearing of the radius valley} \label{sect:void}
As evidenced in Fig.~\ref{fig:fmap}, the radius valley around low mass stars is partially filled by planets.
This feature 

This work elucidates the location of
the radius valley around M dwarf host stars and guides observers to the planetary radii from transit surveys
that are of interest for fully characterizing the radius valley in terms of planetary bulk densities.

mass dependence of the gap: 

The weighted feature radii are also effected by planetary magnetic fields which directly impact the 
efficiency of atmospheric stripping in the photoevaporation scenario \citep{owen19}. The persistence 
of a planetary magnetic field acts to shield the planet's atmosphere from XUV stellar photons thus 
enhancing the retention of the atmosphere and shifting the location of the radius valley to larger 
radii.

valley filling increases with deacreasing stellar mass  

In the photoevaporation scenario, the
partial filling of the gap around low mass stars may be explained by their lower XUV luminosities relative to 
Sun-like such as those included in the CKS stellar sample \citep{}.

This explanation seems to be supported by the stellar mass dependent gap measurements from \cite{fulton18}. 

summary of McDonald+2019 (https://ui-adsabs-harvard-edu.ezp-prod1.hul.harvard.edu/abs/2019ApJ...876...22M/abstract):
X-rays only since XUV observations are difficult for non-Sun-like stars and X-rays are the dominant driver of 
atmospheric loss by photoevaporation. 
Jackson+12 \& Shkolnik+14 derived scalings from data for the LX/Lbol evolution over time for 0.3 - 1.3 solar mass stars
on the MS, low mass stars ($\lesssim 0.8$ M$_{\odot}$) exhibit a LX/Lbol that is typically a few to ten times greater 
than around Sun-like stars ($0.8-1.12$ M$_{\odot}$) (fig 1 in McDonald+2019).
scaling these values by the typical bolometric luminosities of stars in the various mass bins reveals that 
Sun-like stars having higher absolute X-ray luminosities which contributes to more efficient clearing of the 
gap by photoevaporation.

\acknowledgements
We thank Martin Paegert for his efforts in contributing relevent data to and for his assistance in querying the \tess{}
CTL database.


\bibliographystyle{apj}
\bibliography{refs}

\end{document}
