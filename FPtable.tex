\begin{deluxetable}{lccc}
\tabletypesize{\small}
\tablecaption{\ktwo{} false positive rates for small planets around cool stars\label{tab:FP}}
\tablehead{Reference & $N_{\text{FP}}$ & $N_{\text{VP}}$ & FP rate [\%]}
\startdata
\cite{montet15}\tablenotemark{a} & 0 & 8 & $<30.7$ \\
\cite{crossfield16b} & 2 & 39 & $4.9^{+6.0}_{-1.4}$ \\
\cite{dressing17} & 2 & 34 & $5.6^{+6.4}_{-2.0}$ \\
\cite{hirano18}\tablenotemark{a} & 0 & 16 & $<19.5$ \\
\cite{livingston18a}\tablenotemark{a}  & 0 & 14 & $<21.0$ \\
\cite{mayo18}\tablenotemark{b} & 1 & 14 & $6.7^{+12.4}_{-2.0}$
\enddata
\tablecomments{Within each study we only consider PCs with $r_p <4$ R$_{\oplus}$ and orbiting cool stars with \teff{} $<4700$ K. FP: false positive. VP: validated planet.}
\tablenotetext{a}{These studies do not detect any FPs such that the reported FP rate upper limit is represented by its 95\% confidence interval.}
\tablenotetext{b}{\cite{mayo18} did not explicitly classify their non-validated planets as FPs so we define FPs within their sample as any PC whose false positive probability exceeds 10\%.}
\end{deluxetable}
